\documentclass[10pt, a4paper]{article}
\usepackage[top=3cm, bottom=4cm, left=3.5cm, right=3.5cm]{geometry}
\usepackage{amsmath,amsthm,amsfonts,amssymb,amscd, fancyhdr, color, comment, graphicx, environ}
\usepackage{tikz}
\usetikzlibrary{calc}
\usepackage[dvipsnames]{xcolor}
\usepackage{float}
\usepackage{mathtools}
\usepackage{mathrsfs}
\usepackage[math-style=ISO]{unicode-math}
\DeclareSymbolFont{\mathnormal}{letters}
\usepackage{lastpage}

%%%%%%%%%%%%%%%%%%%%%%%%%%%%%%%%%%%%%%%%%%%%%%%%%%%%%%%%%%%%%%%%%%
%%%%%%%%%%%%%%%%%%%%%%%%%%%%%%%%%%%%%%%%%%%%%%%%%%%%%%%%%%%%%%%%%%
%Fill in the appropriate information below
\newcommand{\norm}[1]{\left\lVert#1\right\rVert}     
\newcommand\course{ECON 8010}                            % <-- course name   
\newcommand\hwnumber{7}                                  % <-- homework number
\newcommand\Information{Tate Mason}                        % <-- personal information
%%%%%%%%%%%%%%%%%%%%%%%%%%%%%%%%%%%%%%%%%%%%%%%%%%%%%%%%%%%%%%%%%%
%%%%%%%%%%%%%%%%%%%%%%%%%%%%%%%%%%%%%%%%%%%%%%%%%%%%%%%%%%%%%%%%%%
%Page setup
\pagestyle{fancy}
\headheight 35pt
\lhead{\today}
\rhead{}
\lfoot{}
\pagenumbering{arabic}
\cfoot{\small\thepage}
\rfoot{}
\headsep 1.2em
\renewcommand{\baselinestretch}{1.25}
%%%%%%%%%%%%%%%%%%%%%%%%%%%%%%%%%%%%%%%%%%%%%%%%%%%%%%%%%%%%%%%%%%
%%%%%%%%%%%%%%%%%%%%%%%%%%%%%%%%%%%%%%%%%%%%%%%%%%%%%%%%%%%%%%%%%%
%Add new commands here
\renewcommand{\labelenumi}{\alph{enumi})}
\newcommand{\var}{\text{var}}
\newcommand{\Z}{\mathbb Z}
\newcommand{\R}{\mathbb R}
\newcommand{\Q}{\mathbb Q}
\newcommand{\NN}{\mathbb N}
\newcommand{\PP}{\mathbb P}
\DeclareMathOperator{\Mod}{Mod} 
\renewcommand\lstlistingname{Algorithm}
\renewcommand\lstlistlistingname{Algorithms}
\def\lstlistingautorefname{Alg.}
\newtheorem*{theorem}{Theorem}
\newtheorem*{lemma}{Lemma}
\newtheorem{case}{Case}
\newcommand{\assign}{:=}
\newcommand{\infixiff}{\text{ iff }}
\newcommand{\nobracket}{}
\newcommand{\backassign}{=:}
\newcommand{\tmmathbf}[1]{\ensuremath{\boldsymbol{#1}}}
\newcommand{\tmop}[1]{\ensuremath{\operatorname{#1}}}
\newcommand{\tmtextbf}[1]{\text{{\bfseries{#1}}}}
\newcommand{\tmtextit}[1]{\text{{\itshape{#1}}}}

\newenvironment{itemizedot}{\begin{itemize} \renewcommand{\labelitemi}{$\bullet$}\renewcommand{\labelitemii}{$\bullet$}\renewcommand{\labelitemiii}{$\bullet$}\renewcommand{\labelitemiv}{$\bullet$}}{\end{itemize}}
\catcode`\<=\active \def<{
\fontencoding{T1}\selectfont\symbol{60}\fontencoding{\encodingdefault}}
\catcode`\>=\active \def>{
\fontencoding{T1}\selectfont\symbol{62}\fontencoding{\encodingdefault}}
\catcode`\<=\active \def<{
\fontencoding{T1}\selectfont\symbol{60}\fontencoding{\encodingdefault}}

%%%%%%%%%%%%%%%%%%%%%%%%%%%%%%%%%%%%%%%%%%%%%%%%%%%%%%%%%%%%%%%%%%
%%%%%%%%%%%%%%%%%%%%%%%%%%%%%%%%%%%%%%%%%%%%%%%%%%%%%%%%%%%%%%%%%%
%Begin now!

\begin{document}
  \begin{titlepage}
    \begin{center}
      \vspace*{3cm}
            
        \vspace{1cm}
        \huge
        Homework \hwnumber
            
        \vspace{1.5cm}
        \Large
            
        \textbf{\Information}                      % <-- author
            
        \vfill
        
        A \course \ Homework Assignment
            
        \vspace{1cm}
        \Large

        
        \today
            
    \end{center}
  \end{titlepage}

  \newpage
  \section*{PS4.4}
    \subsection*{Problem}
      "Centipede with a possibly generous player" (see lecture notes) has a unique sequential equilibrium strategy profile in which both players sometimes choose to continue. Does this game possess any other weak sequential equilibrium profiles?
    \subsection*{Solution}
      As found in the lecture, the unique equilibrium in which both players choose to continue is
      \begin{gather*}
        \mu_1(y)=\frac{4}{5};\sigma_2(c)=\frac{4}{19};\sigma_1(c')=\frac{1}{5};\sigma_1(c)=1
      \end{gather*}
      There would be no other weak equilibria as there are no other rational profiles. For instance, if player 1 chooses to stop at $s'$, player 2 would stop at s, thus causing the node not to be reached. If player 1 were to always play $c'$, the probability does not align with beliefs. Thus, there can be no other profiles which are Bayesian rational.
  \section*{Figure 5}
    \begin{center}
      \includegraphics*{fig5.png}
    \end{center}
  \section*{PS4.6}
    \subsection*{Problem}
      Compute all sequential equilirbia of the game in figure 5.
    \subsection*{Solution}
      In all cases, B will not be played as there is a better move for each case a or b. Thus it becomes a game with two brances.
      For Case 1:
      \begin{gather*}
        \sigma_1(O|t_b) = 1 \\
        \sigma_1(I|t_b)=0 \\
        \mu_2(t_b|I) = 0 \\
        \sigma_2(T|I)=1 \\
        \sigma_1(I|t_1)=1
      \end{gather*}
      For Case 2:
      \begin{gather*}
        \sigma_1(I|t_a)=0 \\
        0\geq-\sigma_2(M|I)+2(1-\sigma_2(M|I))\\
        \sigma_2(M|I)\geq\frac{2}{3} \\
        \text{if $\sigma_2(M|I)\in[\frac{2}{3},1)$}:
        \mu_2(t_a|I)=\frac{1}{2}       
      \end{gather*}
      Because player 1 will not play $I$ if type b, player 2 will always play M or $\sigma_2(M|I)=1$.
      Equilibrium:
        \begin{gather*}
          \sigma_2(M|I)=1 \\
          \sigma_1(I|t_b)=0 \\
          \mu_2(t_a|I)\geq\frac{1}{2} \\
        \end{gather*}
  \section*{Figure 6}
    \begin{center}
      \includegraphics*{Fig6.png}
    \end{center}
  \section*{PS4.7}
    \subsection*{Problem}
      Compute all sequential equilibria of the game in figure 6.
    \subsection*{Solution}
      Case 1: 
      \begin{gather*}
        \sigma_1(L|t_a)=\sigma_1(L|t_b)=\sigma_1(L|t_c)=1 \\
        \mu_2(t_a|L) = \mu_2(t_a|R) = 0.6 \\
        \mu_2(t_b|L) = \mu_2(t_b|R) = 0.3 \\
        \mu_2(t_c|L) = \mu_2(t_c|R) = 0.1 \\
        \sigma_2(U|L) = 1 \\
        t_a: \\
        0 > -(\sigma_2(T|R)) + 5(1-\sigma_2(T|R)) \\
        0 > -6\sigma_2(T|R) + 5 \\
        \sigma_2(T|R) \geq \frac{5}{6} \\
        t_b: \\
        0 > 5(\sigma_2(T|R)) - (1-\sigma_2(T|R)) \\
        0 > 6\sigma_2(T|R) - 1 \\
        \sigma_2(T|R) \leq \frac{1}{6} \\
        \therefore \sigma_2(T|R) = \{\alpha|\alpha\in[\frac{1}{6}, \frac{5}{6}]\} \\
      \end{gather*}
      Case 2:
      \begin{gather*}
        \sigma_1(L|t_a) = \sigma_1(L|t_b) = 0; \sigma_1(L|t_c) = 1 \\
        \sigma_1(R|t_a) = \sigma_1(R|t_b) = 1 \\
        \sigma_2(T|R) = \frac{1}{3} \\
        \frac{\sigma_1(R|t_a)}{\sigma_1(R|t_a)+\sigma_1(R|t_b)} = \frac{2}{3} = \mu_2(t_a|R)\\
        \frac{\sigma_1(R|t_b)}{\sigma_1(R|t_a)+\sigma_1(R|t_b)} = \frac{1}{3} = \mu_2(t_b|R) \\
        \mu_2(t_c|L) = 1 \\
        \sigma_2(U|L) = 1 \\
      \end{gather*}
      Case 3:
      \begin{gather*}
        \sigma_1(R|t_a)=\frac{1}{10}; \sigma_1(R|t_b) = 1; \sigma_1(R|t_c) = 0 \\
        \mu_2(t_a|R) = \frac{1}{6} \\
        \mu_2(t_b|R) = \frac{5}{6} \\
        \mu_2(t_a|L) = \mu_2(t_b|L) = \frac{1}{2} \\
        \sigma_2(T|R) = \frac{5}{6} \\
        \sigma_2(U|L) = 1 \\
      \end{gather*}
      Case 4: 
      \begin{gather*}
        \sigma_1(R|t_a) = 1; \sigma_1(R|t_b) = \frac{2}{5}; \sigma_1(R|t_c) = 0 \\
        \mu_2(t_a|R) = \frac{5}{6} \\
        \mu_2(t_b|R) = \frac{1}{6} \\
        \mu_2(t_b|L) = \frac{6}{11} \\
        \mu_2(t_c|L) = \frac{5}{11} \\
        \sigma_2(T|R) = \frac{1}{6} \\
        \sigma_2(U|L) = 1 \\
      \end{gather*}
  \section*{Question 4}
    \subsection*{Problem}
      Consider the following extensive form game of imperfect information played by three venture capital firms $i \in \{1, 2, 3\}$. The game takes place in five periods $t \in \{0, 1, 2, 3, 4\}$ and proceeds as follows.
      \begin{itemize}
          \item At $t = 0$, Nature chooses a state of the world $\theta \in \{H, L\}$ according to the probability distribution $\rho_0$, where
          \[
          \rho_0(H) = \frac{1}{2}, \quad \rho_0(L) = \frac{1}{2}.
          \]
          This state of the world corresponds to whether investing in a certain startup will yield a high payoff (state $H$) or a low payoff (state $L$). It is not observed by any of the firms.
          \item At $t = 1$, Nature chooses a signal $r_1 \in \{h, \ell\}$ according to the probability distribution $\rho(\cdot|\theta)$, where
          \[
          \rho(h|H) = \frac{3}{4}, \quad \rho(\ell|H) = \frac{1}{4}, \quad \rho(h|L) = \frac{1}{4}, \quad \rho(\ell|L) = \frac{3}{4}.
          \]
          This signal is observed by firm 1 (only). Then, firm 1 chooses to either invest ($Y_1$) or not invest ($N_1$). This action is observed by firms 2 and 3.
          \item At $t = 2$, Nature chooses another signal $r_2 \in \{h, \ell\}$ according to $\rho(\cdot|\theta)$. This signal is observed by firm 2 (only). Then, firm 2 chooses to either invest ($Y_2$) or not invest ($N_2$). This action is observed by firm 3.
          \item At $t = 3$, Nature chooses yet another signal $r_3 \in \{h, \ell\}$ according to $\rho(\cdot|\theta)$. This signal is observed by firm 3 (only). Then, firm 3 chooses to either invest ($Y_3$) or not invest ($N_3$).
          \item At $t = 4$, the game ends. Each firm $i$ who chose to invest (i.e., chose action $Y_i$) receives a payoff of 1 if $\theta = H$ and $-\frac{9}{10}$ if $\theta = L$. Each firm who chose not to invest (i.e., chose action $N_i$) receives a payoff of zero, regardless of the state of the world.
      \end{itemize}
      Note that firms’ beliefs are only relevant insofar as they describe the probability placed on each state of the world. Thus, instead of writing beliefs as the probabilities of each decision node (e.g., $\mu_3(H, r_1 = \ell, N_1, r_2 = h, Y_2, r_3 = \ell)$), we can simply write them as the probability of $\theta$ given the current information set (e.g., $\mu_3(H|N_1, Y_2, r_3 = \ell)$).
      \begin{enumerate}
          \item What are the unique Bayesian beliefs $\mu_1(H|r_1 = h)$, $\mu_1(H|r_1 = \ell)$ for player 1 after $r_1 \in \{h, \ell\}$?
          \item In any weak sequential equilibrium, what action will firm 1 take after signal $h$? After $\ell$?
          \item What are the unique Bayesian beliefs
          \[
          \mu_2(H|Y_1, r_2 = h), \quad \mu_2(H|Y_1, r_2 = \ell), \quad \mu_2(H|N_1, r_2 = h), \quad \mu_2(H|N_1, r_2 = \ell)?
          \]
          \item In any weak sequential equilibrium, what action will firm 2 take after $(Y_1, r_2 = h)$? After $(Y_1, r_2 = \ell)$? After $(N_1, r_2 = h)$? After $(N_1, r_2 = \ell)$?
          \item Solve for all weak sequential equilibria. (You only need to solve for beliefs about the state, not about individual decision nodes that follow the same state.)
          \item In any weak sequential equilibrium, what is the probability of a history $(s_1, s_2)$ occurring after which
          \begin{enumerate}
              \item firm 3 invests (i.e., plays $Y_3$), no matter what signal $r_3$ she receives?
              \item firm 3 does not invest (i.e., plays $N_3$), no matter what signal $r_3$ she receives?
              \item firm 3 plays the “correct” action (i.e., $Y_3$ if the state is $H$ or $N_3$ if the state is $L$), no matter what signal $r_3$ she receives?
              \item firm 3 plays the “incorrect” action (i.e., $N_3$ if the state is $H$ or $Y_3$ if the state is $L$), no matter what signal $r_3$ she receives?
          \end{enumerate}
      \end{enumerate}
    \subsection*{Solution}
      \subsubsection*{(a)}
        \begin{gather*}
          \mu_1(H|h) = \frac{\frac{3}{4}\cdot\frac{1}{2}}{(\frac{3}{4}\cdot\frac{1}{2})+(\frac{1}{4}\cdot\frac{1}{2})} \\
          \mu_1(H|h) = \frac{3}{4}\\
          \mu_1(H|\ell) = \frac{\frac{1}{4}\cdot\frac{1}{2}}{(\frac{3}{4}\cdot\frac{1}{2})+(\frac{1}{4}\cdot\frac{1}{2})} \\
          \mu_1(H|\ell) = \frac{1}{4}
        \end{gather*}
      \subsubsection*{(b)}
        \begin{gather*}
          \EE(H|h) = \mu_1(H|h) - (1-\mu_1(H|h))(\frac{9}{10}) \\
          \EE(H|h) = \frac{3}{4} - (\frac{9}{10}\cdot\frac{1}{4}) > 0 \\
          \EE(H|h) = \frac{21}{40} > 0 \therefore \sigma_1(H|h) = Y_1 \\
          \EE(H|\ell) = \mu_1(H|\ell) - (1-\mu_1(H|\ell))(\frac{9}{10}) \\
          \EE(H|\ell) = \frac{1}{4} - \frac{27}{40} \\
          \EE(H|\ell) = -\frac{17}{40} < 0 \therefore \sigma_1(H|\ell) = N_1 \\
        \end{gather*}
      \subsubsection*{(c)}
        \begin{gather*}
          \mu_2(H|Y_1, r_2=h) = \frac{(\frac{3}{4})^2\cdot\frac{1}{2}}{(\frac{3}{4}^2\cdot\frac{1}{2})+(\frac{1}{4}^2\cdot\frac{1}{2})} \\
          \mu_2(H|Y_1, r_2=h) = \frac{9}{10} \\
          \mu_2(H|N_1, r_2=\ell) = \frac{(\frac{1}{4})^2\cdot\frac{1}{2}}{(\frac{1}{4}^2\cdo\frac{1}{2})+(\frac{3}{4}^2\cdot\frac{1}{2})} \\
          \mu_2(H|N_1, r=\ell) = \frac{1}{10} \\
          \mu_2(H|Y_1, r=\ell) = \frac{1}{2} \\
          \mu_2(H|N_1, r=h) = \frac{1}{2}
        \end{gather*}
      \subsubsection*{(d)}
        \begin{gather*}
          \EE(\mu_2(H)) = \mu_2+(1-\mu_2(H))(-\frac{9}{10}) \\
          \mu_2 \geq \frac{9}{19} \\
          \sigma_2(H|Y_1,h) = Y_2 \\
          \sigma_2(H|N_1,h) = Y_2 \\
          \sigma_2(H|Y_1, \ell) = Y_2 \\
          \sigma_2(H|N_1, \ell) = N_2 \\
        \end{gather*}
        All but the case in which player one chooses not to invest when the state is low, player 2 will invest. In the outlier case, they will not invest, following the signal of the player before them. 
      \subsubsection*{(e)}
        \begin{gather*}
          \text{i)} \\
          \mu_3(H|Y_1,Y_2,r_3=h) = Y_3 \\
          \mu_3(H|Y_1,Y_2,r_3=\ell) = Y_3 \\
          \text{(ii)} \\
          \mu_3(H|N_1,Y_2,r_3=h) = Y_3 \\
          \mu_3(H|N_1,Y_2,r_3=\ell) = Y_3 \\
          \text{(iii)} \\
          \mu_3(H|N_1,N_2,r_3=h) = N_3 \\
          \mu_3(H|N_1,N_2,r_3=\ell) = N_3 \\
        \end{gather*}
      \subsubsection*{(f)}
        \begin{gather*}
          \text{(a)} \\
          \frac{1}{2}+(\frac{1}{2})^2 = \frac{3}{4} \\
          \text{(b)} \\
          \frac{1}{2}^2 = \frac{1}{4} \\
          \text{(c)} \\
          (\frac{1}{2}\cdot\frac{3}{4}) + (\frac{1}{2}\cdot\frac{1}{4}) = \frac{1}{2} \\
          \text{(d)} \\
          1 - (f.c) = \frac{1}{2} \\
        \end{gather*}
\end{document}

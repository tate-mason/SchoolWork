\documentclass[10pt,a4paper]{article}
\usepackage[top=3cm,bottom=4cm,left=3.5cm,right=3.5cm]{geometry}
\usepackage{amsmath,amsthm,amsfonts,amssymb,amscd}
\usepackage{fancyhdr,color,comment,graphicx,environ,float,mathtools,mathrsfs}
\newcommand{\norm}[1]{\left\lVert#1\right\rVert}

% Custom headers
\pagestyle{fancy}
\lhead{ECON - 8020}
\chead{}
\rhead{Tate Mason}
\lfoot{}
\cfoot{Assignment 3}
\rfoot{\thepage}

\begin{document}

\title{Assignment 3}
\author{Tate Mason}
\date{Due: February 27th, 11:59pm}
\maketitle

\section*{Question 1: Optimal Auctions (25 pts)}
In this problem, you will compute the auction that maximizes the auctioneer’s expected revenue.

There is a seller looking to sell a single object. There are $N$ bidders (indexed by $i \in \{1, \dots, N\}$) for this object. Each bidder $i$ has an i.i.d value $v_i$ drawn from a distribution $F$ with support on $[0,M]$. We assume that the density function $f$ is continuous, where $F(a) = \int_{0}^{a} f(x)dx$. Furthermore, assume that $\frac{f(x)}{1-F(x)}$ is non-decreasing.

\begin{enumerate}
    \item[(a)] Write down the seller’s optimization problem. Hint: use the revelation principle we discussed in class.
    \item[(b)] Take a given bidder $i$ and the corresponding incentive constraints. Write these constraints as the solution to a maximization problem.
    \item[(c)] Reformulate the constraints using the envelope theorem and derive an expression for $t(v_i)$.
    \item[(d)] Solve the optimization problem for the seller. What is the allocation and transfer rule that maximizes expected revenue?
    \item[(e)] Interpret your answer in (d). What kind of auction is this?
\end{enumerate}

\section*{Question 2 (15 pts)}
A seller is selling a single object. There are $N$ bidders. Each bidder $i$ has an i.i.d value $v_i$ drawn from a distribution $F$ with support $[0,M]$.

Let $OPT(N)$ be the expected revenue from the optimal auction with $N$ bidders. Let $S(N+1)$ denote the expected revenue from a second-price auction with $N+1$ bidders.
\begin{enumerate}
    \item[(a)] Prove that $S(N+1) \geq OPT(N)$.
    \item[(b)] Interpret the result in (a). What does it mean? What is the takeaway?
\end{enumerate}

\section*{Question 3: Correlated Values (10 pts)}
There are two bidders with private values $v_i \in \{1,2\}$. The values are correlated:
\begin{itemize}
    \item Probability $\frac{1}{3}$: both bidders have a value of 1.
    \item Probability $\frac{1}{3}$: both bidders have a value of 2.
    \item Probability $\frac{1}{6}$: one has 2, the other has 1.
\end{itemize}
\begin{enumerate}
    \item[(a)] Suppose the auctioneer runs a second-price auction with random tie-breaking. Is truthful bidding still weakly dominant? Find expected revenue and bidder surplus.
    \item[(b)] Construct an allocation and transfer rule that extracts full surplus while keeping the allocation rule unchanged.
\end{enumerate}

\section*{Question 4 (25 pts)}
This question involves the Principal-Agent problem we examined in Lecture.

The principal (seller) sells a quantity $x \geq 0$ of a good for payment $t$. The cost function is $c(x)$. The agent’s utility is $v(x, \theta) - t$, where $\theta$ is private information.
\begin{enumerate}
    \item[(a)] Write down the seller’s optimization problem (objective, IC, and IR constraints).
    \item[(b)] Rewrite the constraints using ICFOC and monotonicity.
    \item[(c)] Solve for the optimal mechanism (menu/contract).
    \item[(d)] Under what conditions on $v(x,\theta)$ does marginal markup decrease?
    \item[(e)] Show that constant marginal cost implies quantity discounting.
    \item[(f)] Suppose $v(x, \theta) = \theta \gamma(x)$. Show that a power-law distributed $\theta$ leads to a two-part tariff.
\end{enumerate}

\end{document}

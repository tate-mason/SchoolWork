%% ECON 8041 %%
\documentclass[10pt, a4paper]{article}
\usepackage[top=3cm, bottom=4cm, left=3.5cm, right=3.5cm]{geometry}
\usepackage{amsmath, amsthm, amsfonts, amssymb, amscd, fancyhdr, color, comment, graphicx, environ}
\usepackage{float}
\usepackage{mathtools}
\usepackage{mathrsfs}
\newcommand{\course}{ECON - 8040}
\newcommand{\hwnumber}{1}
\newcommand{\Information}{Tate Mason}

\pagestyle{fancy}
\fancyhf{}
\fancyhead[L]{\course}
\fancyhead[C]{Assignment \hwnumber}
\fancyhead[R]{\Information}
\begin{document}

\begin{center}
    \Large \textbf{Assignment \hwnumber} \\
    \Large \textbf{\Information} \\
    \normalsize Due Date: January 23rd, 11:59pm \\
\end{center}

\section*{Question 1: Envelope Theorem}

Consider a constrained optimization problem:
\[
x^*(\theta) = \arg \max_{x \in X(\theta)} f(x, \theta), \quad V(\theta) = \max_{x \in X(\theta)} f(x, \theta)
\]
where $X(\theta) = \{x \in \mathbb{R}^n | g_j(x) \leq \theta_j\}$.

\begin{enumerate}
    \item[(a)] Derive the envelope theorem, computing $\frac{\partial V(\theta)}{\partial \theta_j}$ for each $j \in \{1, \ldots, m\}$.
    \item[(b)] Interpret your answer in (a), especially regarding KKT multipliers.
    \item[(c)] Apply the result to consumer maximization. Define $f(x, \theta)$, constraints $g_j$, and interpret $\theta$ and KKT multipliers.
\end{enumerate}


\section*{Question 2: Topkis' Theorem (Single-Dimension)}

\begin{enumerate}
    \item[(a)] Provide a full proof of Topkis' theorem using the approach from class.
    \item[(b)] Analyze when $q^*(\theta)$ is nondecreasing for a monopolist with inverse demand $p(q)$ and cost $c(q, \theta)$.
    \item[(c)] For a firm minimizing costs with production $f(k, \phi l) = q$, find conditions where optimal $k^*$ is weakly increasing/decreasing in $\phi$.
\end{enumerate}

\section*{Question 3: Topkis' Theorem (Multi-Dimensional)}

A firm has two inputs, capital ($k$) and labor ($l$), that it can use for production. Cost of capital and labor are $r \text{and} w$ , respectively. The firm's production function is 
given by $f(k, l)$, and the firm sells its product at a fixed price $p$. Unlike part (c) in the previous question, the firm is not constrained to produce a fixed level of output.
Thus, the firm's problem is:
\begin{center}
    $\max_{k,l} pf(k,l) - wl - rk$
\end{center}
Assume that $\frac{\partial f}{\partial k} \text{and} \frac{\partial f}{\partial l}$ exist and are positive, and that $\frac{\partial^2 f}{\partial k\partial l} < 0$.
\begin{enumerate}
    \item[(a)] Provide an interpretation for the assumption that $\frac{\partial^2 f}{\partial k\partial l} < 0$.
    \item[(b)] Analyze how optimal labor choice changes as wage $w$ increases.
\end{enumerate}

\section*{Question 4: Putting it All Together}

\begin{enumerate}
    \item[(a)] For utility $u(x, y, z) = x^{1/2} y^{1/2} + z$, show optimum $T$ equals 0 or $W$.
    \item[(b)] For $u(x, y, z) = x^\alpha y^\alpha + z$, derive $T$ in terms of prices and $W$.
    \item[(c)] For $u(x, y, z) = x^\alpha y^\beta + h(z)$, show $T$ is weakly increasing in $W$.
\end{enumerate}

\section*{Solution 1:}
    \subsection*{(a)}
        Using the envelope theorem on the optimization problem:
        \begin{gather*}
        V(\theta)=\max_{x\in\mathcal{X}} f(x,\theta) \\
        \text{where} X(\theta) = \{x\in\mathbb{R}^n|g_j(x)\leq\theta_j\} \\
        \end{gather*}
        \begin{proof}
        \begin{center}
            $\frac{\partial v}{\partial\theta_j} = \frac{\partial f(x^*(\theta),\theta_j)}{\partial\theta_j}
            +\sum\limits_{j=1}^m\frac{\partial f}{\partial x_j}\frac{\partial x^*(\theta_j)}{\partial\theta_j}
            = \frac{\partial f}{\partial\theta_j} + \sum\limits_{j=1}^m\lambda\frac{\partial X(\theta_j)}{\partial x_j}
            \cdot\frac{\partial x^*(\theta)}{\partial\theta_j}$ \\
            $\sum\limits_{j=1}^m\lambda\frac{\partial X(\theta_j)}{\partial x_j}\cdot\frac{\partial x^*(\theta)}{\partial\theta_j}
            = \sum\limits_{j=1}^m\frac{\partial X(\theta)}{\partial(x_j)}\cdot\Delta x_j + \frac{\partial X(\theta)}{\partial\theta_j}\cdot\Delta\theta_j$ \\
        \end{center}
        Now, we can apply the following:
        \begin{center}
            $\frac{1}{\Delta\theta_j}\cdot[\sum\limits_{j=1}^m\frac{\partial X(\theta)}{\partial x_j}\Delta x_j +
            \frac{\partial X(\theta_j)}{\partial\theta_j}]$ \\
            $\sum\limits_{j=1}^m\frac{\partial X(\theta)}{\partial x_j}\frac{\Delta x_j}{\Delta\theta_j} +
            \frac{\partial X(\theta)}{\partial\theta_j}$ \\
        \end{center}
        Finally, we can yield an end result:
        \begin{center}
            $\boxed{\frac{V(\theta)}{\partial\theta_j} = \frac{\partial f}{\partial \theta_j} +\sum\limits_{j=1}^m \lambda_j[-\frac{\partial X(\theta)}{\partial\theta_j}]}$
        \end{center}
        This is the result since, at the max, $x^*=0$ thus eliminating the first term in the summation and leaving the derivative of the
        constraint with respect to $\theta_j$.
        \end{proof}
    \subsection*{(b)}
        The derivative of the value function with respect to $\theta_j$ is the derivative of the objective function with respect to $\theta_j$ plus the KKT multipliers $\sum\limits_{j=1}^m\lambda_j$
        multiplied by the derivative of the constraint with respect to $\theta_j$. This means that the KKT multipliers have an effect on the value function via the constraint. 
    \subsection*{(c)}
        For consumer maximization, the utility function is $f(x,\theta)$, the constraints are $g_j(x)\leq\theta_j$, and $\theta$ is the budget. The KKT multipliers are the shadow prices of the constraints,
        which are the marginal utility of the budget constraint. 
\section*{Solution 2:}
    \subsection*{(a)}
        \begin{proof}
          Let's assume there exist variables $a_H$ and $a_L$ such that $a_H > a_L$. Further, define $f(a_H,\theta') - f(a_L,\theta') \geq f(a_H,\theta)-f(a_L,\theta)$. Next, substitute
          $a_H = \max\{x,x'\} \text{and} a_L = x'$ such that $x\in x^*(\theta) \text{and} x'\in x^*(\theta')$. This yields $f(\max\{x,x'\},\theta') - f(x',\theta') \geq f(\max\{x,x'\},\theta) - f(x',\theta)$.
          Further, define $\theta'>\theta$. If the maximum of $x$ and $x'$ with respect to $\theta$ is $x'$, the right side of the term goes to zero. Otherwise, the inequality holds. Let's
          proceed with the assumption that the right is zeroed out, allowing us the rearrange and get the inequality $f(\max\{x',x\},\theta')\geq f(x',\theta')$. This, therefore, states that 
          $\max\{x',x\} \in x^*(\theta')$. Now, assume that $a_H = x$ and $a_L = \min\{x,x'\}$. This yields $f(x,\theta') - f(\min\{x,x'\},\theta') \geq f(x,\theta) - f(\min\{x,x'\},\theta)$. 
          Now, assuming that $\min\{x,x'\} = x$, the left side equals 0, thus allowing us to rearrange as $f(\min\{x,x'\},\theta)\geq f(x,\theta)$. This result shows that $\min\{x,x'\}\in x^*(\theta)$.
      \end{proof}
    \subsection*{(b)}
      In this case, we can say that $-c^*(q)$ has increasing differences. Derivation follows:
      \begin{gather*}
        p(q)q - c(q,\theta) = \Pi \\
        \text{Applying Topkis' Theorem} \\
        -c(q',\theta')-(-c(q,\theta')) \geq -c(q,\theta)-(-c(q,\theta)) \geq 0 \\
        q\in q^*(\theta); \ q'\in q^*(\theta'); \ \theta' > \theta \\
        c(q,\theta') >_s c(q,\theta)
      \end{gather*}
    \subsection*{(c)}
      \begin{gather*}
        \min_{k,l} rk+wl \ \text{subject to} \ f(k,\phi l) = q \\
        \phi l = L(k,q) \rightarrow l = \frac{L(k,q)}{\phi} \\
        \text{Substitute into the objective function} \\
        \min_{k} rk + w\frac{L(k,q)}{\phi} \\
        g = rk + w\frac{L(k,q)}{\phi} \\
        \frac{\partial g}{\partial k} = r + \frac{w}{\phi}\frac{\partial L(k,q)}{\partial k} = 0 \\
        \frac{\partial^2 g}{\partial k\partial \phi} = -\frac{w}{\phi^2}\frac{\partial L(k,q)}{\partial k} \\
        \text{Discerning $L(k,q)$'s sign} \\
        f(k,\phi\frac{1}{\phi}L(k,q)) \\
        \frac{\partial f}{\partial k}: \ f_k+ f_lL_k(k,q) = 0 \\
        L_k(k,q) = -\frac{f_k}{f_l} \\
      \end{gather*}
      $L_k(k,q) \leq 0 \ \text{if} \ \phi>0$. Since $-\frac{f_k}{f_l}$ is negative since $f_k, f_l\geq0$, 
      the optimal $k$ is non-decreasing in $\phi$ so long as the ratio of marginal product of capital and 
      labor is positive.
      For the case on non-increasing, $k^*$ is non-increasing in $\phi$ if the ratio of the marginal 
      product of capital and labor is negative.

\section*{Question 3:}
    \subsection*{(a)}
      $\frac{\partial^2 f}{\partial l\partial k}<0$ is due to both $\frac{\partial f}{\partial k}, \ \frac{\partial f}{\partial l} >0$. Thus, the cross-partial will be negative due to concavity.
      Further, this makes intuitive sense as an increase in cost in one input would necessitate a decrease in the use of that input to keep costs at a utility maximizing level.
    
    \subsection*{(b)}
      As $w$ increases, the firm will substitute labor for capital as the cost of labor increases. This is due to the given fact that $\frac{\partial^2 f}{\partial l\partial k}<0$ and the firm's objective is to minimize costs.
\section*{Question 4:}
    \subsection*{(a)}
      \begin{gather*}
        {\max_{0\leq T \leq W}} [\max_{p_xx + p_yy = T} x^{\frac{1}{2}}y^{\frac{1}{2}} + \frac{W-T}{p_z}] \\
        \mathcal{L} = (xy)^{\frac{1}{2}} + \lambda(T-p_xx-p_yy)\\
        \mathcal{L}_x: \frac{1}{2}(\frac{y}{x})^{\frac{1}{2}} = \lambda p_x \\
        \mathcal{L}_y: \frac{1}{2}(\frac{x}{y})^{\frac{1}{2}} = \lambda p_y \\
        \frac{y}{x} = \frac{p_x}{p_y} \rightarrow y = \frac{p_x}{p_y}x \\
        T = p_xx + p_y(\frac{p_x}{p_y}x) \rightarrow x = \frac{T}{2p_x} \rightarrow y = \frac{T}{2p_y} \\
        \text{Substitute into the objective function} \\
        \frac{T}{2p_x}^{\frac{1}{2}}\frac{T}{2p_y}^{\frac{1}{2}} + \frac{W-T}{p_z} \\
        \frac{T}{2\sqrt{p_xp_y}} + \frac{W-T}{p_z} \\
        \text{Take the derivative with respect to T} \\
        \boxed{\frac{1}{2\sqrt{p_xp_y}} = \frac{1}{p_z}} \\
      \end{gather*}
      This yields $T = 0$ in the case that $\frac{1}{2\sqrt{p_xp_y}} < \frac{1}{p_z}$ and $T = W$ in the case that $\frac{1}{2\sqrt{p_xp_y}} > \frac{1}{p_z}$.
    \subsection*{(b)}
      \begin{gather*}
        {\max_{0\leq T \leq W}} [\max_{p_xx + p_yy = T} x^{\alpha}y^{\alpha} + \frac{W-T}{p_z}] \\
        \mathcal{L} = (xy)^{\alpha} + \lambda(T-p_xx-p_yy)\\
        \mathcal{L}_x: \alpha x^{\alpha-1}y^{\alpha} = \lambda p_x \\
        \mathcal{L}_y: \alpha x^{\alpha}y^{\alpha-1} = \lambda p_y \\
        \frac{y}{x} = \frac{p_x}{p_y} \rightarrow y = \frac{p_x}{p_y}x \\
        T = p_xx + p_y(\frac{p_x}{p_y}x) \rightarrow x = \frac{T}{2p_x} \rightarrow y = \frac{T}{2p_y} \\
        \text{Substitute into the objective function} \\
        \frac{T}{(4p_xp_y)^{\alpha}}^{2\alpha} + \frac{W-T}{p_z} \\
        \text{Take the derivative with respect to T} \\
        (2\alpha)\frac{T}{4p_xp_y}^{2\alpha-1} = \frac{1}{p_z} \\
        \text{Solve for T} \\
        \boxed{T = (\frac{(4p_xp_y)^{\alpha}}{2\alpha p_z})^{\frac{1}{2\alpha-1}}}
      \end{gather*}
      $T$ does not rely on $W$ due to the utility function having quasilinearity.
    \subsection*{(c)}
      \begin{gather*}
        {\max_{0\leq T \leq W}} [\max_{p_xx + p_yy = T} x^{\alpha}y^{\beta} + h(\frac{W-T}{p_z})] \\
        \mathcal{L} = x^{\alpha}y^{\beta} + h(\frac{W-T}{p_z}) + \lambda(T-p_xx-p_yy)\\
        \mathcal{L}_x: \alpha x^{\alpha-1}y^{\beta} = \lambda p_x \\
        \mathcal{L}_y: \beta x^{\alpha}y^{\beta-1} = \lambda p_y \\
        \frac{y}{x}\frac{\alpha}{\beta} = \frac{p_x}{p_y} \rightarrow y = \frac{\beta}{\alpha}\frac{p_x}{p_y}x \\
        T = p_xx + p_y(\frac{\beta}{\alpha}\frac{p_x}{p_y}x) \rightarrow x = \frac{\alpha T}{p_x(\alpha+\beta)} \rightarrow y = \frac{\beta T}{p_y(\alpha+\beta)} \\
        \text{Substitute into the objective function} \\
        \left(\frac{\alpha T}{p_x(\alpha+\beta)}\right)^{\alpha}\left(\frac{\beta T}{p_y(\alpha+\beta)}\right)^{\beta} + h(\frac{W-T}{p_z}) \\
        \text{Let's Simplify a bit:} \\
        \left(\frac{(\alpha\beta)T}{(\alpha+\beta)p_xp_y}\right)^{\alpha+\beta} + h(\frac{W-T}{p_z}) \\
        \gamma(T) = \left(\frac{(\alpha\beta)T}{(\alpha+\beta)p_xp_y}\right)^{\alpha+\beta} \\
        u(T,W) = \gamma(T) + h(\frac{W-T}{p_z}) \\
        \text{Supermodularity of $u(T,W)$ implies that $T$ is weakly increasing in $W$} \\
        \gamma(T) \ \text{does not depend on $W$ and thus is irrelevant} \\
        h(z) \ \text{depends on $W$ and thus should be analyzed:} \\
        \exists \ W'>W ; \ T'>T \\
        u(T',W') - u(T, W') \geq u(T',W) - u(T,W) \\
        \text{Holding $T$ constant,} \ h(\frac{W'-T}{p_z}) \geq h(\frac{W-T}{p_z}) \\
        \text{Holding $W$ constant,} \ h(\frac{W-T'}{p_z}) \leq h(\frac{W-T}{p_z}) \\
        \therefore u(T',W') - u(T, W') \geq u(T',W) - u(T,W) \ \text{is satisfied}\\
      \end{gather*}
      By increasing differences being satisfied, $T^*$ is weakly increasing in $W$.
\end{document}
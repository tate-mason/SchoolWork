\documentclass[10pt,a4paper]{article}
\usepackage[top=3cm,bottom=4cm,left=3.5cm,right=3.5cm]{geometry}
\usepackage{amsmath,amsthm,amsfonts,amssymb,amscd}
\usepackage{fancyhdr,color,comment,graphicx,environ,float,mathtools,mathrsfs,bbm}
\newcommand{\norm}[1]{\left\lVert#1\right\rVert}

% Custom headers
\pagestyle{fancy}
\lhead{ECON - 8050}
\chead{}
\rhead{Tate Mason}
\lfoot{}
\cfoot{Homework 3}
\rfoot{\thepage}

\begin{document}

\title{Homework 3}
\author{ECON 8050: Macroeconomics II \\ Tate Mason}
\date{}
\maketitle

\section*{Problem 1: Dynamic Programming}
Consider the following model with a disability shock. There are three sources of uncertainty:
\begin{itemize}
    \item Out-of-pocket medical shock evolving according to transition matrix $\Psi(x_t|x_{t-1})$.
    \item Productivity evolving according to $T(z_t|z_{t-1})$.
    \item Disability shock.
\end{itemize}

The timing of events is as follows: At the beginning of the period, an individual with savings $k_t$ learns their productivity $z_t$ and medical shock $x_t$. Then they decide whether to work ($l_t = 0$ or $l_t > 0$). If working, labor income is $w z_t l_t$. Then, they decide about consumption $c_t$ and savings $k_{t+1}$.

At the end of the period, the disability shock is realized with probability $d$. Disabled individuals stay permanently disabled, do not work, receive constant benefits $DI$, and make only consumption/savings decisions. Medical spending for disabled individuals is fully covered by public insurance.

\begin{enumerate}
    \item[(1)] Write down the dynamic programming problem of a non-disabled individual, denoting the value function as $V_t$.
    \item[(2)] Write down the dynamic programming problem of a disabled individual, denoting the value function as $V_t^d$.
    \item[(3)] Modify the problem assuming disabled individuals can recover with probability $f$. Recovered individuals draw new productivity realizations from the invariant distribution.
    \item[(4)] Extend the model to allow non-disabled individuals to falsely claim disability benefits, introducing the value function for falsely disabled $V_t^{fd}$.
\end{enumerate}

\section*{Problem 2: Consumption-Savings Model}
A consumer with infinite life maximizes quadratic utility:
\begin{align*}
    u(c_t) &= -\frac{1}{2} (c_t - \bar{c})^2
\end{align*}
where future utility is discounted at rate $\beta$ and borrowing/savings occur at interest rate $r$ with $\beta(1+r) = 1$.

The consumer’s endowment $y_t$ is i.i.d. with values $y_H$ and $y_L$ occurring with probabilities $p_H$ and $p_L$ respectively. The budget constraint is:
\begin{align*}
    c_t = a_t(1+r) + y_t - a_{t+1}.
\end{align*}

\begin{enumerate}
    \item[(1)] Solve for the consumption and saving functions. Provide intuition on when savings are positive or negative.
    \item[(2)] Introduce a borrowing constraint $a_{t+1} \geq 0$. Solve the consumer’s problem in recursive form numerically using given parameters.
    \item[(3)] Plot policy functions $a_{t+1}$ and $c_t$ as functions of current assets $a_t$ for cases with and without borrowing constraints.
    \item[(4)] Simulate the income process and optimal decision rules over $T=100$ periods. Compare results with and without borrowing constraints.
\end{enumerate}

\section*{Solution 1}
  \subsection*{Part (i)}
    The Bellman for an able bodied person with probability of becoming disabled $d$ is as follows:
    \begin{gather*}
      V_t(k_t,x_t,z_t) = {\max\atop{c_t,l_t,k_{t+1}}}\{u(c_t,l_t) + 
      \beta(1-d)\sum\limits_{x_t}\sum\limits_{z_t}\Psi(x_t|x_{t-1})
      \text{T}(z_t|z_{t-1})V_{t+1}(k_{t+1},x_{t+1},z_{t+1}) \\
      + \beta d\sum\limits_{x_t}\Psi(x_t|x_{t-1})V_{t+1}(k_{t+1},x_{t+1},z_{t+1})\} \\
      \shortintertext{s.t.} \\
      c_t + k_{t+1} + x_t = wz_tl_t + k_t(1+r) \\
    \end{gather*}
  \subsection*{Part (ii)}
    The Bellman for an individual who is disabled and has no probability of recovery can be represented as follows:
    \begin{gather*}
      V_t^d(k_t,x_t) = {\max\atop{c_t,k_{t+1}}}\{u(c_t,0) + \beta\sum
      \limits_{x_t}\Psi(x_t|x_{t-1})V^d_{t+1}(k_{t+1},x_{t+1})\} \\
      \shortintertext{s.t.} \\
      c_t + k_{t+1} = DI + k_t(1+r)
    \end{gather*}
  \subsection*{Part (iii)}
    The Bellman equation for an individual who is disabled but has a probability of recovery is as follows:
    \begin{gather*}
      V_t^{df}(k_t,x_t) = {\max\atop{c_t,k_{t+1}}}\{u(c_t,l_t) +\beta 
      f\sum\limits_{x_t|x_{t-1}}\sum\limits_{z_t|z_{t-1}}\Psi(x_t|x_
      {t-1})\text{T}(z_t|z_{t-1})V^{d}_{t+1}(k_{t+1}, x_{t+1}, z_{t+1}) \\
      + \beta(1-f)\sum\limits_{x_t}\Psi(x_t|x_{t-1})V^d_{t+1}(k_{t+1},x_{t+1})\} \\
      \shortintertext{s.t.} \\
      c_t + k_{t+1} + x_t = DI + k_t(1+r) \\
    \end{gather*}
  \subsection*{Part (iv)}
    Finally, the Bellman for someone who has the option to fake disability is as follows:
    \begin{gather*}
      V_t^{df}(k_t,x_t,z_t) = {\max\atop{c_t,k_{t+1},l_t}}\{u(c_t,l_t) +\beta 
      f\sum\limits_{x_t|x_{t-1}}\sum\limits_{z_t|z_{t-1}}\Psi(x_t|x_
      {t-1})\text{T}(z_t|z_{t-1})V^{d}_{t+1}(k_{t+1}, x_{t+1}, z_{t+1}) \\
      + \beta(1-f)\sum\limits_{x_t}\Psi(x_t|x_{t-1})V^d_{t+1}(k_{t+1},x_{t+1}) \\
      + \beta f \mathbbm{1}_{fake = 1} \sum_{x_t}(x_t|x_{t-1})V_{t+1}^{fd}(k_{t+1},x_{t+1})\} \\
      \shortintertext{s.t.} \\
      c_t + k_{t+1} + x_t = wz_tl_t + DI\mathbbm{1}_{D = 1 or fake = 1} + k_t(1+r)
    \end{gather*}
\section*{Solution 2:} 
  \subsection{Part (i)}
    \begin{gather*}
      \sum\limits_{t=0}^{\infty}\beta^t[-\frac{1}{2}(c_t -
      \bar{c})^2] \\
      \shortintertext{s.t.} \\
      c_t + a_{t+1} = (1+r)a_t + y_t \\
      \shortintertext{Dynamic Programming is as follows:} \\
      V(a_t,y_t) = {\max\atop{c_t,a_{t+1}}}\{-\frac{1}{2}(c_t - \bar{c})^2 + \beta\mathbb{E}_t[V(a_{t+1},y_{t+1})]\} \\
      \shortintertext{Because the utility is quadratic and $\beta(1+r) = 1$, we know the Euler is as follows:} \\
      c_t = \mathbb{E}_t[c_{t+1}] \\
      \shortintertext{Plugging in from the constraint,} \\
      c_t = \mathbb{E}[y_t] + (1+r)\mathbb{E}[a_t] - \mathbb{E}[a_{t+1}] \\
      \shortintertext{Expectations can be represented as follows:}
      \mathbb{E}[y_t] = p_Hy_H + p_Ly_L \\
      \mathbb{E}[a_{t+1}] = a_t \ \text{smoothes consumption due to instability of income} \\
      \shortintertext{From the FOC w.r.t to $a_{t+1}$} \\
      c_t - \bar{c} = \mathbb{E}_t[c_{t+1}-\bar{c}] \\
      c_t = \bar{c} + (1+r)a_t + y_t - a_t \\
      c_t = \bar{c} + ra_t + y_t \\
      \shortintertext{This is the consumption function when there is no borrowing constraint and infinite lifetime. Let's find the saving function below.} \\
      a_{t+1} = (1+r)a_t + y_t - c_t \\
      a_{t+1} = (1+r)a_t + y_t - (\bar{c}+ra_t+y_t) \\
      a_{t+1} = a_t - \bar{c} \\
      \shortintertext{This is the optimal savings function in the case of no borrowing constraint and infinite lifetime.}
    \end{gather*}
    The savings function states that depending on the consumer's state of assets, they will either save or borrow. This depends on the relationship between $\bar{c}$ and $a_t$. When $a_t>\bar{c}$, the agent will consume more as they do not have to worry about their savings. In the opposite case, they will save more, trying to replenish their savings.
\end{document}

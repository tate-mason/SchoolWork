\documentclass[10pt,a4paper]{article}
\usepackage[top=3cm,bottom=4cm,left=3.5cm,right=3.5cm]{geometry}
\usepackage{amsmath,amsthm,amsfonts,amssymb,amscd,fancyhdr,color,comment,graphicx,environ,float,mathtools,mathrsfs,unicode-math}

% Custom Commands
\newcommand{\norm}[1]{\left\lVert#1\right\rVert}

% Document Information
\title{Homework 2}
\author{Tate Mason}
\date{ECON - 8050}

\begin{document}

\maketitle

\section*{Problem 1: Costs of Business Cycle}
  Let utility be given by:
  \begin{equation*}
      E^{-1} \sum_{t=0}^{\infty} \beta^t U(c_t)
  \end{equation*}
  where the utility function is CRRA:
  \begin{equation*}
      U(c_t) = \frac{c_t^{1-\gamma}}{1- \gamma}
  \end{equation*}

  The consumption process is
  \begin{equation*}
      c_t = c_{t-1}^{\alpha} \varepsilon_t \exp(\mu)
  \end{equation*}
  where
  \begin{equation*}
      \mu = \frac{-\sigma_{\varepsilon}^2 (1-\alpha)}{2 (1-\alpha^2)}, \quad \log \varepsilon_t \sim N(0, \sigma_{\varepsilon}^2) \text{ and i.i.d.}
  \end{equation*}
  Thus, the log of consumption follows an AR(1) process:
  \begin{equation*}
      \log c_t = \mu + \alpha \log c_{t-1} + \log \varepsilon_t
  \end{equation*}

  \subsection*{Part A}
    Find the unconditional mean of $c_t$, $E(c_t)$. (Hint: recall the properties of the lognormal distribution).

  \subsection*{Part B}
    Define lifetime utility before any uncertainty is realized as:
    \begin{equation*}
        V_0 = E^{-1} \sum_{t=0}^{\infty} \beta^t U(c_t)
    \end{equation*}
    Assume $c_0$ is drawn from the invariant (unconditional) distribution of $c$.
    Now define:
    \begin{equation*}
        V(\lambda) = E^{-1} \sum_{t=0}^{\infty} \beta^t U [c_t(1 + \lambda)]
    \end{equation*}
    This is lifetime utility when every period consumption is increased by $(1+\lambda)$. Express $V(\lambda)$ as a function of $\mu, \sigma_{\varepsilon}^2, \alpha, \gamma, \beta$.

  \subsection*{Part C}
    Denote $V_0$ as the lifetime utility when $c_t$ is deterministic and equal to its unconditional mean found in part A). Find the compensation $\lambda$ such that $V(\lambda) = V_0$. Find how much compensation the consumer has to be given in order to be indifferent between the stochastic and deterministic cases, Provide economic intuition.

  \subsection*{Part D}
    Denote the interest rate as $r$. Find consumption $c_t$.

\section*{Problem 2: Non-Expected Utility Framework}
  This problem follows the Kreps and Porteus (1978), Epstein and Zin (1991), and Weil (1990) frameworks.

  Let remaining lifetime utility at time $t$, once $c_t$ is known, be given by $v_t$, satisfying:
  \begin{equation}
      v_t = \left[ (1-\beta) c_t^{\rho} + \beta (E_t v_{t+1}^{\alpha})^{\frac{\rho}{\alpha}} \right]^{\frac{1}{\rho}}
  \end{equation}
  where $1-\alpha$ represents risk aversion and $1-\rho$ represents the inverse of the intertemporal elasticity of substitution. In standard expected utility, $\alpha = \rho$.

  Denote pre-realization lifetime utility at time $t$ as $U_t$, where:
  \begin{equation*}
      U_t = (E_t v_t^{\alpha})^{\frac{1}{\alpha}}
  \end{equation*}

  \subsection*{Part A}
    Prove that multiplying $c_t$ by $\lambda$ for all $t = 0, 1, \dots, \infty$ is equivalent to multiplying $v_t$ by $\lambda$.
    (Hint: start by assuming this holds, substitute into equation (1), and show $v_t$ scales linearly.)

  \subsection*{Part B}
    Suppose for all $t$, we replace $c_t$ with a deterministic constant $\bar{c} = E[c_t]$. Compare welfare in this case with uncertain $c_t$. Specifically, find $\eta$ such that multiplying $c_t$ by $(1+ \eta)$ makes ex-ante welfare $U_0$ equal to that in the deterministic case. Express $\eta$ in terms of $U_0$ and $\bar{c}$.

  \subsection*{Part C}
    Suppose consumption follows one of two sequences: with probability $\frac{1}{2}$, $c_t = c_l$ for all $t$, and with probability $\frac{1}{2}$, $c_t = c_h$ for all $t$. The sequence is revealed at $t = 0$. Find $\eta$ and analyze its dependence on $\rho$ and $\alpha$.

  \subsection*{Part D}
    Now assume $c_t$ is i.i.d., where each period $c_t = c_l$ with probability $\frac{1}{2}$ and $c_h$ with probability $\frac{1}{2}$.
    \begin{enumerate}
        \item Derive an implicit equation for $U_0$.
        \item Analyze whether $\eta$ depends on $\alpha$ and $\rho$.
    \end{enumerate}

  \subsection*{Part E}
    Solve for $U_0$ numerically using Matlab with given parameters: $\beta = 0.95$, $c_l = e^{0.98}$, $c_h = e^{1.02}$. Compute $\eta$ for:
    \begin{itemize}
        \item $\alpha = 1, 0.5, -1$
        \item $\rho = 1, 0.5, -1$
    \end{itemize}
    Report results in a table and provide economic intuition.
    (Hint: Use an iterative approach to solve $U_0 = f(U_0)$ until convergence with tolerance $10^{-8}$.)
\section{Solution 1}
  \subsection*{(A)}
    
\end{document}

\documentclass[10pt,a4paper]{article}
\usepackage[top=3cm,bottom=4cm,left=3.5cm,right=3.5cm]{geometry}
\usepackage{amsmath,amsthm,amsfonts,amssymb,amscd}
\usepackage{fancyhdr,color,comment,graphicx,environ,float,mathtools,mathrsfs,bbm,listings}
\newcommand{\norm}[1]{\left\lVert#1\right\rVert}

% Custom headers
\pagestyle{fancy}
\lhead{ECON - 8050}
\chead{}
\rhead{Tate Mason}
\lfoot{}
\cfoot{Homework 4}
\rfoot{\thepage}

\begin{document}

\title{Homework 4}
\author{ECON 8050: Macroeconomics II \\ Tate Mason}
\date{}
\maketitle

The process for $y = log(income)$ is:
\begin{align*}
    y_{t+1} = \mu + \rho y_t + \sigma \varepsilon_{t+1}
\end{align*}
where $\varepsilon \sim N(0, 1)$

\begin{enumerate}
    \item[(1)] Set $\mu = 0$, $\rho = 0.9$ and $\sigma = 0.0242$. Discretize the process for $y$ with $9$ points. Download the Matlab code ghquad.m to compute Gauss-Hermit grids and weights. Use $10,000$ as maxit input. As an output, print out the vector of discretized $y$ and the transition matrix.
    
    \item[(2)] Simulate the Markov chain and compute the implied autocorrelation coefficient ($\hat{\rho}$). Note: use 1 million observations to simulate a persistent AR process. Disregard first 1000 observations. Report both $\hat{\rho}$ and $\hat{\sigma}$ computed from the simulated data.
\end{enumerate}
\end{document}

\documentclass[10pt, a4paper]{article}
\usepackage[top=3cm, bottom=4cm, left=3.5cm, right=3.5cm]{geometry}
\usepackage{amsmath,amsthm,amsfonts,amssymb,amscd, fancyhdr, color, comment, graphicx, environ}
\usepackage{float}
\usepackage{mathtools}
\usepackage{mathrsfs}
\usepackage[math-style=ISO]{unicode-math}
\DeclareSymbolFont{\mathnormal}{letters}
\usepackage{lastpage}

%%%%%%%%%%%%%%%%%%%%%%%%%%%%%%%%%%%%%%%%%%%%%%%%%%%%%%%%%%%%%%%%%%
%%%%%%%%%%%%%%%%%%%%%%%%%%%%%%%%%%%%%%%%%%%%%%%%%%%%%%%%%%%%%%%%%%
%Fill in the appropriate information below
\newcommand{\norm}[1]{\left\lVert#1\right\rVert}     
\newcommand\course{ECON - 8040}                            
\newcommand\lecnumber{1 - 01/06/2025}                                 
\newcommand\Information{Tate Mason}                        
%%%%%%%%%%%%%%%%%%%%%%%%%%%%%%%%%%%%%%%%%%%%%%%%%%%%%%%%%%%%%%%%%%
%%%%%%%%%%%%%%%%%%%%%%%%%%%%%%%%%%%%%%%%%%%%%%%%%%%%%%%%%%%%%%%%%%
%Page setup
\pagestyle{fancy}
\headheight 35pt
\lhead{\today}
\rhead{}
\lfoot{}
\pagenumbering{arabic}
\cfoot{\small\thepage}
\rfoot{}
\headsep 1.2em
\renewcommand{\baselinestretch}{1.25}

%%%%%%%%%%%%%%%%%%%%%%%%%%%%%%%%%%%%%%%%%%%%%%%%%%%%%%%%%%%%%%%%%%
%%%%%%%%%%%%%%%%%%%%%%%%%%%%%%%%%%%%%%%%%%%%%%%%%%%%%%%%%%%%%%%%%%
%Add new commands here
\renewcommand{\labelenumi}{\arabic{enumi}.}
\newcommand{\Z}{\mathbb Z}
\newcommand{\R}{\mathbb R}
\newcommand{\Q}{\mathbb Q}
\newcommand{\NN}{\mathbb N}
\newcommand{\PP}{\mathbb P}
\newcommand{\sumt}{$\sum\limits_{t=0}^{\infty}$}
\DeclareMathOperator{\Mod}{Mod}
\newtheorem*{theorem}{Theorem}
\newtheorem*{lemma}{Lemma}
\newcommand{\assign}{:=}

\begin{document}
  \begin{titlepage}
    \begin{center}
      \vspace*{3cm}
            
      \vspace{1cm}
      \huge
      Lecture \lecnumber
    \end{center}
  \end{titlepage}
  \section*{Markov Process}
  AR(1):
  \begin{gather*}
    z_{t+1} = \rho z_t + \epsilon_{t+1} \ \text{s.t.} \ \epsilon_{t+1}\sim N(0,\sigma_{\epsilon}^2)
  \end{gather*}
  limited memory process, allows for us to have an educated guess despite not possessing full knowledge
  \section*{Markov Chain}
  A Markov Chain is a Markov Process with a finite number of values
  \begin{gather*}
    z\in\{z_1,z_2,...,z_n\}
  \end{gather*}
  Objectives:
  \begin{enumerate}
  \item set of states: 
  \begin{gather*}
    \{z_1,z_2,...,z_n\}
  \end{gather*}
  \item transition matrix - probability of transition between states
  \begin{gather*}
    i,j\in\{1,...,n\}\Rightarrow p_{i,j} = \text{Prob}(z_{t+1}=z_j|z_t = z_i)
  \end{gather*}
  \item Unconditional Distribution $\Pi_t$
  \begin{gather*}
  \Pi_{t+1}' = \Pi_t'\cdot p \\
  (\Pi_{1t+1},...,\Pi_{nt+1}) = (\Pi_{1t},...,\Pi_{nt})\begin{pmatrix}
    p_{11} & ... & p_{1n} \\
    ... & ... & ... \\
    p_{1n} & ... & p_{nn}
    \end{pmatrix}
  \end{gather*}
  P is unconditional probability that new state is reached in future
  \end{enumerate}
\end{document}

  

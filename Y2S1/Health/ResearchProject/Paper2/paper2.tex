\documentclass[11pt,a4paper]{article}

% Essential packages
\usepackage[utf8]{inputenc}
\usepackage[T1]{fontenc}
\usepackage[margin=1in]{geometry}
\usepackage{amsmath,amsfonts,amssymb}
\usepackage{graphicx}
\usepackage{booktabs}
\usepackage{url}
\usepackage{hyperref}
\usepackage{natbib}
\usepackage{setspace}


% Additional useful packages
\usepackage{fancyhdr}
\usepackage{subcaption}
\usepackage{tikz}
\usepackage{algorithm}
\usepackage{algorithmic}
\usepackage{listings}
\usepackage{xcolor}

% Configure hyperref
\hypersetup{
    colorlinks=true,
    linkcolor=blue,
    filecolor=magenta,      
    urlcolor=cyan,
    citecolor=red,
    pdftitle={Your Paper Title},
    pdfauthor={Your Name},
}

% Set line spacing
\onehalfspacing

% Configure headers and footers
\pagestyle{fancy}
\fancyhf{}
\rhead{\thepage}
\lhead{\textit{Your Paper Title}}
\renewcommand{\headrulewidth}{0.4pt}

% Title page information
\title{The Price of Comfort: Investigating Willingness to Pay for \\
Medically Assisted Suicide}

\author{
    Tate Mason\thanks{Tate.Mason@uga.edu} \\
    \textit{John Munro Godfrey Sr. Dept. of Economics} \\
    \textit{University of Georgia} \\
    \textit{Athens, Georgia} \\
}

\date{\today}

% Abstract environment
\newenvironment{abstract}%
{\cleardoublepage\null \vfill\begin{center}%
\bfseries \abstractname \end{center}}%
{\vfill\null}

\begin{document}

% Title page
\maketitle
\thispagestyle{empty}

% Abstract
\begin{abstract}
\noindent % Write a concise summary (150-250 words) covering: research problem, methodology, key findings, and significance. State your main contributions clearly.

\vspace{0.3cm}
\noindent \textbf{Keywords:} % List 3-6 keywords relevant to your research
\end{abstract}

\newpage
\setcounter{page}{1}

% Main content
\section{Introduction}
\label{sec:introduction}
% Provide a broad introduction to your field and gradually narrow to your specific research question.
% Motivate why this work is necessary and important.

\subsection{Background and Motivation}
\label{subsec:background}

In recent decades, mental health has become a focal point of health research. This is warranted given the spike in diagnoses of diseases like depression, anxiety, or other more severe
afflictions. Cases of suicide have also grown to a massive degree, with people feeling the need to take their own life as an escape from their disorder. Traditionally, this has been a
private event, which is seen as an act of emotional desperation. However, in recent years, countries like Spain, Portual, and Canada have implemented medically assisted suicide policies
which seek to grant agency over death to those with incurable diseases or otherwise those in tremendous pain with no hope of alleviation.
% Provide context for your research area. Explain why this problem is important and worth investigating.

\subsection{Problem Statement}
\label{subsec:problem}

These policies raise interesting questions. Of course, morals are of concern to some, but in an economic sense there are key factors which warrant research. Of interest for this paper,
what is the willingness to pay of the terminally ill to put an end to their life if a policy like this was put into place in the United States? This kind of procedure, of course, requires
pharmaceutical resources, and thus would come at cost. Beyond the landscape of implemented policy, it would be interesting to determine the moral hazard of a policy like this. For instance,
if a person with severe depression and chronic suicidal ideation engaging in riskier behavior as they pursue death by either risk or medically assisted suicide.
% Clearly articulate the specific problem or research question you're addressing. Be precise and focused.

\subsection{Contributions}
\label{subsec:contributions}

% List your main contributions clearly and concisely:
% - What novel methods, insights, or results do you provide?
% - How do they advance the state of the art?
% - What specific problems do they solve?

\subsection{Paper Organization}
\label{subsec:organization}

% Briefly outline how the paper is structured. Guide readers through your narrative flow.

\section{Related Work}
\label{sec:related_work}

% Comprehensively review relevant prior work. Organize thematically, not chronologically.
% Show how your work builds on, differs from, or improves upon existing approaches.

\subsection{Previous Approaches}
\label{subsec:previous_approaches}

% Survey existing methods and approaches relevant to your problem.
% Group similar approaches and explain their key ideas, strengths, and weaknesses.

\subsection{Limitations of Existing Work}
\label{subsec:limitations}

% Identify specific gaps, limitations, or open problems in existing approaches.
% This motivates why your work is needed and positions your contributions.

\section{Methodology}
\label{sec:methodology}

% Describe your approach in sufficient detail for reproducibility.
% Include mathematical formulations, algorithms, and theoretical analysis as needed.

\subsection{Problem Formulation}
\label{subsec:formulation}

% Formally define the problem you're solving using mathematical notation.
% Establish assumptions, constraints, and objectives clearly.

\subsection{Proposed Approach}
\label{subsec:approach}

% Detail your methodology including:
% - Core algorithms and their rationale
% - Mathematical derivations and proofs
% - Implementation details crucial for reproducibility
% - Complexity analysis if relevant

\section{Experimental Evaluation}
\label{sec:experiments}

% Present your experimental design, results, and analysis.
% Be objective and thorough in reporting both positive and negative results.

\subsection{Experimental Setup}
\label{subsec:setup}


\subsection{Datasets}
\label{subsec:datasets}


\subsection{Results}
\label{subsec:results}

% Present results clearly and objectively:
% - Use tables and figures effectively
% - Report statistical significance where appropriate
% - Include error bars or confidence intervals
% - Compare against all relevant baselines
% - Report both quantitative metrics and qualitative observations

\subsection{Analysis}
\label{subsec:analysis}

% Provide deeper analysis of your results:
% - Statistical significance tests
% - Ablation studies showing contribution of different components
% - Error analysis and failure cases
% - Computational complexity and runtime analysis
% - Sensitivity analysis for key parameters

\section{Discussion}
\label{sec:discussion}

% Interpret your findings, discuss broader implications, and acknowledge limitations honestly.

\subsection{Interpretation of Results}
\label{subsec:interpretation}

% Explain what your results mean in the context of your research questions:
% - How do findings relate to your hypotheses?
% - What insights do the results provide?
% - How do they compare to previous work?
% - What are the practical implications?

\subsection{Limitations}
\label{subsec:limitations}

% Honestly discuss limitations of your work:
% - Methodological limitations
% - Dataset limitations or biases
% - Assumptions that may not hold in practice
% - Scope limitations
% - Computational or resource constraints

\subsection{Future Work}
\label{subsec:future_work}

% Suggest concrete directions for future research:
% - How could your approach be extended or improved?
% - What new research questions arise from your work?
% - How could limitations be addressed?
% - What are the next logical steps?

\section{Conclusion}
\label{sec:conclusion}

% Summarize your main contributions, key findings, and their significance.
% Avoid simply repeating the abstract - provide a synthesis that reinforces your contributions.


\section*{Acknowledgments}

% Thank people who contributed to the work but aren't co-authors.
% Acknowledge funding sources, institutional support, or resources used.

% Bibliography
\bibliographystyle{plainnat}
\bibliography{references} % Make sure you have a references.bib file

% Appendices (if needed)
\appendix
\section{Additional Experimental Results}
\label{app:additional_results}

% Include supplementary results that support your main findings but are too detailed for the main text:
% - Extended result tables
% - Additional ablation studies
% - Results on additional datasets
% - Detailed parameter sensitivity analysis

\section{Mathematical Proofs}
\label{app:proofs}

% Include detailed mathematical proofs and derivations:
% - Theoretical analysis too lengthy for the main text
% - Convergence proofs
% - Complexity analysis details
% - Complete derivations of key equations

\end{document}

\documentclass[11pt,a4paper]{article}

% Essential packages
\usepackage[utf8]{inputenc}
\usepackage[T1]{fontenc}
\usepackage[margin=1in]{geometry}
\usepackage{amsmath,amsfonts,amssymb}
\usepackage{graphicx}
\usepackage{booktabs}
\usepackage{url}
\usepackage{hyperref}
\usepackage{natbib}
\usepackage{setspace}


% Additional useful packages
\usepackage{fancyhdr}
\usepackage{subcaption}
\usepackage{tikz}
\usepackage{algorithm}
\usepackage{algorithmic}
\usepackage{listings}
\usepackage{xcolor}

% Configure hyperref
\hypersetup{
    colorlinks=true,
    linkcolor=blue,
    filecolor=magenta,      
    urlcolor=cyan,
    citecolor=red,
    pdftitle={Your Paper Title},
    pdfauthor={Your Name},
}

% Set line spacing
\onehalfspacing

% Configure headers and footers
\pagestyle{fancy}
\fancyhf{}
\rhead{\thepage}
\lhead{\textit{Your Paper Title}}
\renewcommand{\headrulewidth}{0.4pt}

% Title page information
\title{Drug Search and Physician Hazard: \\
An Investigation into Addict Behavior and Policy Remedies}

\author{
    Tate Mason\thanks{Tate.Mason@uga.edu} \\
    \textit{University of Georgia} \\
    \textit{Athens, Georgia} \\
}

\date{\today}

% Abstract environment
\newenvironment{abstract}%
{\cleardoublepage\null \vfill\begin{center}%
\bfseries \abstractname \end{center}}%
{\vfill\null}

\begin{document}

% Title page
\maketitle
\thispagestyle{empty}

% Abstract
\begin{abstract}
\noindent % Write a concise summary (150-250 words) covering: research problem, methodology, key findings, and significance. State your main contributions clearly.

\vspace{0.3cm}
\noindent \textbf{Keywords:} % List 3-6 keywords relevant to your research
\end{abstract}

\newpage
\setcounter{page}{1}

% Main content
\section{Introduction}
\label{sec:introduction}

Over the last few decades, prescription drug abuse has become a significant and growing public health concern across the globe. The misuse of prescription drugs, specifially opioids, benzodiazepines, and stimulants, has led to the question of how to mitigate the practice of
physician search. Physician search refers to the practice of patients seeking multiple doctors in the hope of "scoring" a presription to continue their addiction. While there are preventative measures in place, like registries of offenders, the practice persists. Coninciding
with this question, it would be of great use to ascertain the incentive for physicians to enable the misuse of drugs, gaining a repeat source of revenue.

This paper seeks to understand the interplay of addicts and physician search, in hope of policy remedies which are more effective than a list. Further, the analysis of prescriber responsibility is also of great importance. Being able to understand how morals and fiscal
incentives contrast in this case could help to implement policy which negates the opportunity for prescribers to mis-prescribe a sensitive drug. Finally, this paper will look to gain an insight into potential rehabilitation remedies which may help take the onus off both
addicts and physicians.

% Provide a broad introduction to your field and gradually narrow to your specific research question.
% Motivate why this work is necessary and important.

\subsection{Background and Motivation}
\label{subsec:background}

% Provide context for your research area. Explain why this problem is important and worth investigating.

\subsection{Problem Statement}
\label{subsec:problem}

% Clearly articulate the specific problem or research question you're addressing. Be precise and focused.

\subsection{Contributions}
\label{subsec:contributions}

% List your main contributions clearly and concisely:
% - What novel methods, insights, or results do you provide?
% - How do they advance the state of the art?
% - What specific problems do they solve?

\subsection{Paper Organization}
\label{subsec:organization}

% Briefly outline how the paper is structured. Guide readers through your narrative flow.

\section{Related Work}
\label{sec:related_work}

% Comprehensively review relevant prior work. Organize thematically, not chronologically.
% Show how your work builds on, differs from, or improves upon existing approaches.

\subsection{Previous Approaches}
\label{subsec:previous_approaches}

% Survey existing methods and approaches relevant to your problem.
% Group similar approaches and explain their key ideas, strengths, and weaknesses.

\subsection{Limitations of Existing Work}
\label{subsec:limitations}

% Identify specific gaps, limitations, or open problems in existing approaches.
% This motivates why your work is needed and positions your contributions.

\section{Methodology}
\label{sec:methodology}

% Describe your approach in sufficient detail for reproducibility.
% Include mathematical formulations, algorithms, and theoretical analysis as needed.

\subsection{Problem Formulation}
\label{subsec:formulation}

% Formally define the problem you're solving using mathematical notation.
% Establish assumptions, constraints, and objectives clearly.

\subsection{Proposed Approach}
\label{subsec:approach}

% Detail your methodology including:
% - Core algorithms and their rationale
% - Mathematical derivations and proofs
% - Implementation details crucial for reproducibility
% - Complexity analysis if relevant

\section{Experimental Evaluation}
\label{sec:experiments}

% Present your experimental design, results, and analysis.
% Be objective and thorough in reporting both positive and negative results.

\subsection{Experimental Setup}
\label{subsec:setup}

Describe your experimental methodology, including:
\begin{itemize}
    \item Datasets used
    \item Evaluation metrics
    \item Baseline methods
    \item Implementation details
    \item Hardware/software specifications
\end{itemize}

\subsection{Datasets}
\label{subsec:datasets}

Provide details about the datasets used in your evaluation.

\begin{table}[htbp]
\centering
\caption{Dataset Statistics}
\label{tab:datasets}
\begin{tabular}{@{}lrrr@{}}
\toprule
Dataset & Training Samples & Test Samples & Features \\
\midrule
Dataset 1 & 10,000 & 2,500 & 784 \\
Dataset 2 & 50,000 & 10,000 & 3,072 \\
Dataset 3 & 1,000,000 & 100,000 & 128 \\
\bottomrule
\end{tabular}
\end{table}

\subsection{Results}
\label{subsec:results}

% Present results clearly and objectively:
% - Use tables and figures effectively
% - Report statistical significance where appropriate
% - Include error bars or confidence intervals
% - Compare against all relevant baselines
% - Report both quantitative metrics and qualitative observations

\subsection{Analysis}
\label{subsec:analysis}

% Provide deeper analysis of your results:
% - Statistical significance tests
% - Ablation studies showing contribution of different components
% - Error analysis and failure cases
% - Computational complexity and runtime analysis
% - Sensitivity analysis for key parameters

\section{Discussion}
\label{sec:discussion}

% Interpret your findings, discuss broader implications, and acknowledge limitations honestly.

\subsection{Interpretation of Results}
\label{subsec:interpretation}

% Explain what your results mean in the context of your research questions:
% - How do findings relate to your hypotheses?
% - What insights do the results provide?
% - How do they compare to previous work?
% - What are the practical implications?

\subsection{Limitations}
\label{subsec:limitations}

% Honestly discuss limitations of your work:
% - Methodological limitations
% - Dataset limitations or biases
% - Assumptions that may not hold in practice
% - Scope limitations
% - Computational or resource constraints

\subsection{Future Work}
\label{subsec:future_work}

\end{document}

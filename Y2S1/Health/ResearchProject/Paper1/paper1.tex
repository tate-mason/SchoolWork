\documentclass[11pt,a4paper]{article}

% Essential packages
\usepackage[utf8]{inputenc}
\usepackage[T1]{fontenc}
\usepackage[margin=1in]{geometry}
\usepackage{amsmath,amsfonts,amssymb}
\usepackage{graphicx}
\usepackage{booktabs}
\usepackage{url}
\usepackage{hyperref}
\usepackage{natbib}
\usepackage{setspace}


% Additional useful packages
\usepackage{fancyhdr}
\usepackage{subcaption}
\usepackage{tikz}
\usepackage{algorithm}
\usepackage{algorithmic}
\usepackage{listings}
\usepackage{xcolor}

% Configure hyperref
\hypersetup{
    colorlinks=true,
    linkcolor=blue,
    filecolor=magenta,      
    urlcolor=cyan,
    citecolor=red,
    pdftitle={GLP-1 and Moral Hazard: Encouraging Obesity},
    pdfauthor={Tate Mason},
}

% Set line spacing
\onehalfspacing

% Configure headers and footers
\pagestyle{fancy}
\fancyhf{}
\rhead{\thepage}
\lhead{\textit{GLP-1 and Moral Hazard: Encouraging Obesity}}
\renewcommand{\headrulewidth}{0.4pt}

% Title page information
\title{GLP-1 and Moral Hazard: Encouraging Obesity}

\author{
    Tate Mason\thanks{Corresponding author: your.email@university.edu} \\
    \textit{John Munro Godfrey Sr. Dept. of Economics} \\
    \textit{University of Georgia} \\
}

\date{\today}

% Abstract environment
\newenvironment{abstract}%
{\cleardoublepage\null \vfill\begin{center}%
\bfseries \abstractname \end{center}}%
{\vfill\null}

\begin{document}

% Title page
\maketitle
\thispagestyle{empty}

% Abstract
\begin{abstract}
\noindent % Write a concise summary (150-250 words) covering: research problem, methodology, key findings, and significance. State your main contributions clearly.

\vspace{0.3cm}
\noindent \textbf{Keywords:} % List 3-6 keywords relevant to your research
\end{abstract}

\newpage
\setcounter{page}{1}

% Main content
\section{Introduction}
\label{sec:introduction}

% Provide a broad introduction to your field and gradually narrow to your specific research question.
% Motivate why this work is necessary and important.

\subsection{Background and Motivation}
\label{subsec:background}

In the last few years, a classification of drugs known as GLP-1 receptor agonists have become massively prevalent. The original purpose
of these drugs was to help type-2 diabetics with management of blood sugar. Particularly, the drug aimed to assist those with 
cardiovascular problems or obesity. However, the drug has gained broader popularity via its appetite suppressing and weight loss properties. Those without diabetes
are being prescribed the drug both with cause and without. Whether it be helping obese individuals begin to lose weight, or act
assisting people who want to lose a few pounds, the different variants of GLP-1 agonists have become wildly popular. 

The popularity of the drug brings to mind a question regarding how it effects people's consumption. Specifically, are those who
take it for cosmetic reasons making the corresponding lifestyle improvements, or does the drug simply allow them to consume with
a diminished risk of weight gain? Further, if the drug does encourage lifestyle improvements through consumption, are there policy
remedies to more broadly utilize it to tackle the obesity problem in the United States?

The growth in prescribed use of the GLP-1 drugs has also allowed for massive growth in the companies which made it to the game early is staggering.
For instance, Novo Nordisk, manufacturers of Wegovy, saw 30\percent year-on-year growth from 2024-2025. Understanding how the specific medications
have influenced growth for pharmaceutical manufacturers would allow for insights into the larger market.
% Provide context for your research area. Explain why this problem is important and worth investigating.

\subsection{Problem Statement}
\label{subsec:problem}
This paper seeks to understand the effect of GLP-1 drugs on the consumption habits of individuals who were prescribed the drug
for cosmetic reasons. Particularly, gaining an understanding of how consumption of goods like fast food, alcohol, or exercise 
was altered upon receiving a prescription. Gaining this understanding would allow for an understanding of if the medicine helps
to foster broader lifestyle changes, or is more akin to offsetting consumption of unhealthy food. While difficulties may arise in
the differences in access by income, looking into this question would allow for valuable economic and policy insights.
% Clearly articulate the specific problem or research question you're addressing. Be precise and focused.

\subsection{Contributions}
\label{subsec:contributions}

% List your main contributions clearly and concisely:
% - What novel methods, insights, or results do you provide?
% - How do they advance the state of the art?
% - What specific problems do they solve?

\subsection{Paper Organization}
\label{subsec:organization}

% Briefly outline how the paper is structured. Guide readers through your narrative flow.

\section{Related Work}
\label{sec:related_work}

% Comprehensively review relevant prior work. Organize thematically, not chronologically.
% Show how your work builds on, differs from, or improves upon existing approaches.

\subsection{Previous Approaches}
\label{subsec:previous_approaches}

% Survey existing methods and approaches relevant to your problem.
% Group similar approaches and explain their key ideas, strengths, and weaknesses.

\subsection{Limitations of Existing Work}
\label{subsec:limitations}

% Identify specific gaps, limitations, or open problems in existing approaches.
% This motivates why your work is needed and positions your contributions.

\section{Methodology}
\label{sec:methodology}

% Describe your approach in sufficient detail for reproducibility.
% Include mathematical formulations, algorithms, and theoretical analysis as needed.

\subsection{Problem Formulation}
\label{subsec:formulation}

% Formally define the problem you're solving using mathematical notation.
% Establish assumptions, constraints, and objectives clearly.

\subsection{Proposed Approach}
\label{subsec:approach}

% Detail your methodology including:
% - Core algorithms and their rationale
% - Mathematical derivations and proofs
% - Implementation details crucial for reproducibility
% - Complexity analysis if relevant

\section{Experimental Evaluation}
\label{sec:experiments}

% Present your experimental design, results, and analysis.
% Be objective and thorough in reporting both positive and negative results.

\subsection{Experimental Setup}
\label{subsec:setup}

Describe your experimental methodology, including:
\begin{itemize}
    \item Datasets used
    \item Evaluation metrics
    \item Baseline methods
    \item Implementation details
    \item Hardware/software specifications
\end{itemize}

\subsection{Datasets}
\label{subsec:datasets}

Provide details about the datasets used in your evaluation.

\begin{table}[htbp]
\centering
\caption{Dataset Statistics}
\label{tab:datasets}
\begin{tabular}{@{}lrrr@{}}
\toprule
Dataset & Training Samples & Test Samples & Features \\
\midrule
Dataset 1 & 10,000 & 2,500 & 784 \\
Dataset 2 & 50,000 & 10,000 & 3,072 \\
Dataset 3 & 1,000,000 & 100,000 & 128 \\
\bottomrule
\end{tabular}
\end{table}

\subsection{Results}
\label{subsec:results}

% Present results clearly and objectively:
% - Use tables and figures effectively
% - Report statistical significance where appropriate
% - Include error bars or confidence intervals
% - Compare against all relevant baselines
% - Report both quantitative metrics and qualitative observations

\subsection{Analysis}
\label{subsec:analysis}

% Provide deeper analysis of your results:
% - Statistical significance tests
% - Ablation studies showing contribution of different components
% - Error analysis and failure cases
% - Computational complexity and runtime analysis
% - Sensitivity analysis for key parameters

\section{Discussion}
\label{sec:discussion}

% Interpret your findings, discuss broader implications, and acknowledge limitations honestly.

\subsection{Interpretation of Results}
\label{subsec:interpretation}

% Explain what your results mean in the context of your research questions:
% - How do findings relate to your hypotheses?
% - What insights do the results provide?
% - How do they compare to previous work?
% - What are the practical implications?

\subsection{Limitations}
\label{subsec:limitations}

% Honestly discuss limitations of your work:
% - Methodological limitations
% - Dataset limitations or biases
% - Assumptions that may not hold in practice
% - Scope limitations
% - Computational or resource constraints

\subsection{Future Work}
\label{subsec:future_work}

% Suggest concrete directions for future research:
% - How could your approach be extended or improved?
% - What new research questions arise from your work?
% - How could limitations be addressed?
% - What are the next logical steps?

\section{Conclusion}
\label{sec:conclusion}

% Summarize your main contributions, key findings, and their significance.
% Avoid simply repeating the abstract - provide a synthesis that reinforces your contributions.

In this paper, we presented [brief summary of contribution]. Our experimental evaluation demonstrates [key findings]. The implications of this work include [broader impact]. Future research directions include [future work].

\section*{Acknowledgments}

% Thank people who contributed to the work but aren't co-authors.
% Acknowledge funding sources, institutional support, or resources used.

% Bibliography
\bibliographystyle{plainnat}
\bibliography{references} % Make sure you have a references.bib file

% Appendices (if needed)
\appendix
\section{Additional Experimental Results}
\label{app:additional_results}

% Include supplementary results that support your main findings but are too detailed for the main text:
% - Extended result tables
% - Additional ablation studies
% - Results on additional datasets
% - Detailed parameter sensitivity analysis

\section{Mathematical Proofs}
\label{app:proofs}

% Include detailed mathematical proofs and derivations:
% - Theoretical analysis too lengthy for the main text
% - Convergence proofs
% - Complexity analysis details
% - Complete derivations of key equations

\end{document}

% Options for packages loaded elsewhere
% Options for packages loaded elsewhere
\PassOptionsToPackage{unicode}{hyperref}
\PassOptionsToPackage{hyphens}{url}
\PassOptionsToPackage{dvipsnames,svgnames,x11names}{xcolor}
%
\documentclass[
  letterpaper,
  DIV=11,
  numbers=noendperiod]{scrartcl}
\usepackage{xcolor}
\usepackage{amsmath,amssymb}
\setcounter{secnumdepth}{-\maxdimen} % remove section numbering
\usepackage{iftex}
\ifPDFTeX
  \usepackage[T1]{fontenc}
  \usepackage[utf8]{inputenc}
  \usepackage{textcomp} % provide euro and other symbols
\else % if luatex or xetex
  \usepackage{unicode-math} % this also loads fontspec
  \defaultfontfeatures{Scale=MatchLowercase}
  \defaultfontfeatures[\rmfamily]{Ligatures=TeX,Scale=1}
\fi
\usepackage{lmodern}
\ifPDFTeX\else
  % xetex/luatex font selection
\fi
% Use upquote if available, for straight quotes in verbatim environments
\IfFileExists{upquote.sty}{\usepackage{upquote}}{}
\IfFileExists{microtype.sty}{% use microtype if available
  \usepackage[]{microtype}
  \UseMicrotypeSet[protrusion]{basicmath} % disable protrusion for tt fonts
}{}
\makeatletter
\@ifundefined{KOMAClassName}{% if non-KOMA class
  \IfFileExists{parskip.sty}{%
    \usepackage{parskip}
  }{% else
    \setlength{\parindent}{0pt}
    \setlength{\parskip}{6pt plus 2pt minus 1pt}}
}{% if KOMA class
  \KOMAoptions{parskip=half}}
\makeatother
% Make \paragraph and \subparagraph free-standing
\makeatletter
\ifx\paragraph\undefined\else
  \let\oldparagraph\paragraph
  \renewcommand{\paragraph}{
    \@ifstar
      \xxxParagraphStar
      \xxxParagraphNoStar
  }
  \newcommand{\xxxParagraphStar}[1]{\oldparagraph*{#1}\mbox{}}
  \newcommand{\xxxParagraphNoStar}[1]{\oldparagraph{#1}\mbox{}}
\fi
\ifx\subparagraph\undefined\else
  \let\oldsubparagraph\subparagraph
  \renewcommand{\subparagraph}{
    \@ifstar
      \xxxSubParagraphStar
      \xxxSubParagraphNoStar
  }
  \newcommand{\xxxSubParagraphStar}[1]{\oldsubparagraph*{#1}\mbox{}}
  \newcommand{\xxxSubParagraphNoStar}[1]{\oldsubparagraph{#1}\mbox{}}
\fi
\makeatother

\usepackage{color}
\usepackage{fancyvrb}
\newcommand{\VerbBar}{|}
\newcommand{\VERB}{\Verb[commandchars=\\\{\}]}
\DefineVerbatimEnvironment{Highlighting}{Verbatim}{commandchars=\\\{\}}
% Add ',fontsize=\small' for more characters per line
\usepackage{framed}
\definecolor{shadecolor}{RGB}{241,243,245}
\newenvironment{Shaded}{\begin{snugshade}}{\end{snugshade}}
\newcommand{\AlertTok}[1]{\textcolor[rgb]{0.68,0.00,0.00}{#1}}
\newcommand{\AnnotationTok}[1]{\textcolor[rgb]{0.37,0.37,0.37}{#1}}
\newcommand{\AttributeTok}[1]{\textcolor[rgb]{0.40,0.45,0.13}{#1}}
\newcommand{\BaseNTok}[1]{\textcolor[rgb]{0.68,0.00,0.00}{#1}}
\newcommand{\BuiltInTok}[1]{\textcolor[rgb]{0.00,0.23,0.31}{#1}}
\newcommand{\CharTok}[1]{\textcolor[rgb]{0.13,0.47,0.30}{#1}}
\newcommand{\CommentTok}[1]{\textcolor[rgb]{0.37,0.37,0.37}{#1}}
\newcommand{\CommentVarTok}[1]{\textcolor[rgb]{0.37,0.37,0.37}{\textit{#1}}}
\newcommand{\ConstantTok}[1]{\textcolor[rgb]{0.56,0.35,0.01}{#1}}
\newcommand{\ControlFlowTok}[1]{\textcolor[rgb]{0.00,0.23,0.31}{\textbf{#1}}}
\newcommand{\DataTypeTok}[1]{\textcolor[rgb]{0.68,0.00,0.00}{#1}}
\newcommand{\DecValTok}[1]{\textcolor[rgb]{0.68,0.00,0.00}{#1}}
\newcommand{\DocumentationTok}[1]{\textcolor[rgb]{0.37,0.37,0.37}{\textit{#1}}}
\newcommand{\ErrorTok}[1]{\textcolor[rgb]{0.68,0.00,0.00}{#1}}
\newcommand{\ExtensionTok}[1]{\textcolor[rgb]{0.00,0.23,0.31}{#1}}
\newcommand{\FloatTok}[1]{\textcolor[rgb]{0.68,0.00,0.00}{#1}}
\newcommand{\FunctionTok}[1]{\textcolor[rgb]{0.28,0.35,0.67}{#1}}
\newcommand{\ImportTok}[1]{\textcolor[rgb]{0.00,0.46,0.62}{#1}}
\newcommand{\InformationTok}[1]{\textcolor[rgb]{0.37,0.37,0.37}{#1}}
\newcommand{\KeywordTok}[1]{\textcolor[rgb]{0.00,0.23,0.31}{\textbf{#1}}}
\newcommand{\NormalTok}[1]{\textcolor[rgb]{0.00,0.23,0.31}{#1}}
\newcommand{\OperatorTok}[1]{\textcolor[rgb]{0.37,0.37,0.37}{#1}}
\newcommand{\OtherTok}[1]{\textcolor[rgb]{0.00,0.23,0.31}{#1}}
\newcommand{\PreprocessorTok}[1]{\textcolor[rgb]{0.68,0.00,0.00}{#1}}
\newcommand{\RegionMarkerTok}[1]{\textcolor[rgb]{0.00,0.23,0.31}{#1}}
\newcommand{\SpecialCharTok}[1]{\textcolor[rgb]{0.37,0.37,0.37}{#1}}
\newcommand{\SpecialStringTok}[1]{\textcolor[rgb]{0.13,0.47,0.30}{#1}}
\newcommand{\StringTok}[1]{\textcolor[rgb]{0.13,0.47,0.30}{#1}}
\newcommand{\VariableTok}[1]{\textcolor[rgb]{0.07,0.07,0.07}{#1}}
\newcommand{\VerbatimStringTok}[1]{\textcolor[rgb]{0.13,0.47,0.30}{#1}}
\newcommand{\WarningTok}[1]{\textcolor[rgb]{0.37,0.37,0.37}{\textit{#1}}}

\usepackage{longtable,booktabs,array}
\usepackage{calc} % for calculating minipage widths
% Correct order of tables after \paragraph or \subparagraph
\usepackage{etoolbox}
\makeatletter
\patchcmd\longtable{\par}{\if@noskipsec\mbox{}\fi\par}{}{}
\makeatother
% Allow footnotes in longtable head/foot
\IfFileExists{footnotehyper.sty}{\usepackage{footnotehyper}}{\usepackage{footnote}}
\makesavenoteenv{longtable}
\usepackage{graphicx}
\makeatletter
\newsavebox\pandoc@box
\newcommand*\pandocbounded[1]{% scales image to fit in text height/width
  \sbox\pandoc@box{#1}%
  \Gscale@div\@tempa{\textheight}{\dimexpr\ht\pandoc@box+\dp\pandoc@box\relax}%
  \Gscale@div\@tempb{\linewidth}{\wd\pandoc@box}%
  \ifdim\@tempb\p@<\@tempa\p@\let\@tempa\@tempb\fi% select the smaller of both
  \ifdim\@tempa\p@<\p@\scalebox{\@tempa}{\usebox\pandoc@box}%
  \else\usebox{\pandoc@box}%
  \fi%
}
% Set default figure placement to htbp
\def\fps@figure{htbp}
\makeatother





\setlength{\emergencystretch}{3em} % prevent overfull lines

\providecommand{\tightlist}{%
  \setlength{\itemsep}{0pt}\setlength{\parskip}{0pt}}



 


\KOMAoption{captions}{tableheading}
\usepackage{rotating}
\makeatletter
\@ifpackageloaded{caption}{}{\usepackage{caption}}
\AtBeginDocument{%
\ifdefined\contentsname
  \renewcommand*\contentsname{Table of contents}
\else
  \newcommand\contentsname{Table of contents}
\fi
\ifdefined\listfigurename
  \renewcommand*\listfigurename{List of Figures}
\else
  \newcommand\listfigurename{List of Figures}
\fi
\ifdefined\listtablename
  \renewcommand*\listtablename{List of Tables}
\else
  \newcommand\listtablename{List of Tables}
\fi
\ifdefined\figurename
  \renewcommand*\figurename{Figure}
\else
  \newcommand\figurename{Figure}
\fi
\ifdefined\tablename
  \renewcommand*\tablename{Table}
\else
  \newcommand\tablename{Table}
\fi
}
\@ifpackageloaded{float}{}{\usepackage{float}}
\floatstyle{ruled}
\@ifundefined{c@chapter}{\newfloat{codelisting}{h}{lop}}{\newfloat{codelisting}{h}{lop}[chapter]}
\floatname{codelisting}{Listing}
\newcommand*\listoflistings{\listof{codelisting}{List of Listings}}
\makeatother
\makeatletter
\makeatother
\makeatletter
\@ifpackageloaded{caption}{}{\usepackage{caption}}
\@ifpackageloaded{subcaption}{}{\usepackage{subcaption}}
\makeatother
\usepackage{bookmark}
\IfFileExists{xurl.sty}{\usepackage{xurl}}{} % add URL line breaks if available
\urlstyle{same}
\hypersetup{
  pdftitle={Reduced Form Coding Assignment - ECON 8250},
  pdfauthor={Tate Mason},
  colorlinks=true,
  linkcolor={blue},
  filecolor={Maroon},
  citecolor={Blue},
  urlcolor={Blue},
  pdfcreator={LaTeX via pandoc}}


\title{Reduced Form Coding Assignment - ECON 8250}
\author{Tate Mason}
\date{}
\begin{document}
\maketitle


\begin{enumerate}
\def\labelenumi{\arabic{enumi}.}
\tightlist
\item
  Start by coming up with a research question where you might use this
  research design. This does not need to be too creative, I just want an
  example. However, I would prefer if it wasn't the one I used in class
  and was vaguely related to health. Intuitively and in words, describe
  what the endogeneity concern might be with a question like this.
\item
  Tell me basics about the dataset you simulate. What is your unit of
  observation? Then, in words, describe what variables you are assuming
  constitute the ``true model'' and which variables you are assuming you
  can observe and cannot observe. Describe any important correlations
  between variables. Also, describe other variables, like policies (for
  diff-in-diff), thresholds (for RD), or instruments (for IV). Give me
  an equation for your ``true model'' and introduce all the letters you
  are using. I want an equation, written like they would be written in a
  paper, not STATA code. Then separately, tell me what your ``true''
  coefficients are (i.e.~β = 2).
\item
  In words and equations, describe the regressions you are running. Both
  the regressions that have an endogeneity problem and the ones which
  you ``fix.''
\item
  Produce a table of summary statistics with the mean, standard
  deviation, number of observations, min and max of each variable you
  use. This is both regressors and outcome variables. You do not need to
  show me summary statistics for fixed effects.
\item
  Produce regression results in nice table layout, with intuitive
  variable labels (i.e.~not stata variable names), and not too many
  variables (i.e.~don't display fixed effects). Describe the regression
  results for each of your regressions in words.
\end{enumerate}

\section{Fixed Effects Model}\label{fixed-effects-model}

\subsection{1. Research Question}\label{research-question}

How does insurance premium rise with activity and risk preference?

\subsection{2.}\label{section}

\begin{Shaded}
\begin{Highlighting}[]
\FunctionTok{set.seed}\NormalTok{(}\DecValTok{0219}\NormalTok{)}
\NormalTok{n }\OtherTok{\textless{}{-}} \DecValTok{1000}
\NormalTok{id }\OtherTok{\textless{}{-}} \FunctionTok{rep}\NormalTok{(}\DecValTok{1}\SpecialCharTok{:}\NormalTok{n, }\AttributeTok{each =} \DecValTok{50}\NormalTok{)}
\NormalTok{time }\OtherTok{\textless{}{-}} \FunctionTok{rep}\NormalTok{(}\DecValTok{1}\SpecialCharTok{:}\DecValTok{50}\NormalTok{, }\AttributeTok{times =}\NormalTok{ n)}
\NormalTok{activity\_level }\OtherTok{\textless{}{-}} \FunctionTok{rnorm}\NormalTok{(n }\SpecialCharTok{*} \DecValTok{50}\NormalTok{, }\AttributeTok{mean =} \DecValTok{3}\NormalTok{, }\AttributeTok{sd =} \DecValTok{1}\NormalTok{) }\CommentTok{\# hours of activity per week}
\NormalTok{risk\_pref }\OtherTok{\textless{}{-}} \FunctionTok{rbinom}\NormalTok{(n }\SpecialCharTok{*} \DecValTok{50}\NormalTok{, }\DecValTok{1}\NormalTok{, }\FloatTok{0.5}\NormalTok{) }\CommentTok{\# 0 = low risk preference, 1 = high risk preference}
\NormalTok{insprem }\OtherTok{\textless{}{-}} \DecValTok{200} \SpecialCharTok{+} \DecValTok{5} \SpecialCharTok{*}\NormalTok{ activity\_level }\SpecialCharTok{+} \DecValTok{20} \SpecialCharTok{*}\NormalTok{ risk\_pref }
\SpecialCharTok{+} \FunctionTok{rnorm}\NormalTok{(n, }\AttributeTok{mean =} \DecValTok{0}\NormalTok{, }\AttributeTok{sd =} \DecValTok{10}\NormalTok{)}
\NormalTok{data }\OtherTok{\textless{}{-}} \FunctionTok{data.frame}\NormalTok{(}\AttributeTok{id =}\NormalTok{ id, }\AttributeTok{activity\_level =}\NormalTok{ activity\_level, }\AttributeTok{risk\_pref =}\NormalTok{ risk\_pref, }\AttributeTok{insprem =}\NormalTok{ insprem)}
\end{Highlighting}
\end{Shaded}

The unit of observation is an individual with 1000 agents. Each agent
has 50 observations over time. In this model, \(Y\) is insurance
premium, \(X\) is activity level, and \(W\) is risk preference. The true
model is:

\[InsurancePremium_i = \beta_0 + \beta_1\cdot ActivityLevel_i + \beta_2\cdot RiskPreference_i + \epsilon_i,\]

where \(InsurancePremium_i\) is the insurance premium for individual i,
\(ActivityLevel_i\) is the activity level of individual i,
\(RiskPreference_i\) is the risk preference of individual i, and
\(\epsilon_i\) is the error term. The true coefficients are:
\(\beta_0 = 200\), \(\beta_1 = 5\), \(\beta_2 = 20\), and
\(\sigma = 10\). We observe \(InsurancePremium_i\), \(ActivityLevel_i\),
and \(RiskPreference_i\).

\subsection{3. Regressions}\label{regressions}

The regression with endogeneity problem is:

\[InsurancePremium_i = \alpha_0 + \alpha_1\cdot ActivityLevel_i + u_i,\]

where \(u_i\) is the error term which includes the risk preference
parameter. The regression that ``fixes'' the endogeneity problem using
fixed effects is:

\[InsurancePremium_{it} = \gamma_0 + \gamma_1\cdot ActivityLevel_{it} + \gamma_2\cdot RiskPreference_{it} + v_{it},\]

where \(InsurancePremium_{it}\) is the insurance premium for individual
i at time t, \(ActivityLevel_{it}\) is the activity level of individual
i at time t, \(RiskPreference_{it}\) is the risk preference of
individual i at time t, and \(v_{it}\) is the error term.

\subsection{4. Summary statistics}\label{summary-statistics}

\begin{Shaded}
\begin{Highlighting}[]
\FunctionTok{library}\NormalTok{(stargazer)}
\end{Highlighting}
\end{Shaded}

\begin{verbatim}

Please cite as: 
\end{verbatim}

\begin{verbatim}
 Hlavac, Marek (2022). stargazer: Well-Formatted Regression and Summary Statistics Tables.
\end{verbatim}

\begin{verbatim}
 R package version 5.2.3. https://CRAN.R-project.org/package=stargazer 
\end{verbatim}

\begin{Shaded}
\begin{Highlighting}[]
\FunctionTok{stargazer}\NormalTok{(data[, }\SpecialCharTok{{-}}\FunctionTok{c}\NormalTok{(}\DecValTok{1}\NormalTok{, }\DecValTok{2}\NormalTok{)], }
  \AttributeTok{title =} \StringTok{"Summary Statistics"}\NormalTok{, }
  \AttributeTok{type =} \StringTok{"latex"}\NormalTok{, }\AttributeTok{digits =} \DecValTok{2}\NormalTok{,}
  \AttributeTok{covariate.labels =} \FunctionTok{c}\NormalTok{(}\StringTok{"Insurance Premium"}\NormalTok{, }\StringTok{"Risk Preference"}\NormalTok{),}
  \AttributeTok{summary.stat =} \FunctionTok{c}\NormalTok{(}\StringTok{"n"}\NormalTok{, }\StringTok{"mean"}\NormalTok{, }\StringTok{"sd"}\NormalTok{, }\StringTok{"min"}\NormalTok{, }\StringTok{"max"}\NormalTok{))}
\end{Highlighting}
\end{Shaded}

\% Table created by stargazer v.5.2.3 by Marek Hlavac, Social Policy
Institute. E-mail: marek.hlavac at gmail.com \% Date and time: Thu, Sep
04, 2025 - 13:00:16

\begin{table}[!htbp] \centering 
  \caption{Summary Statistics} 
  \label{} 
\begin{tabular}{@{\extracolsep{5pt}}lccccc} 
\\[-1.8ex]\hline 
\hline \\[-1.8ex] 
Statistic & \multicolumn{1}{c}{N} & \multicolumn{1}{c}{Mean} & \multicolumn{1}{c}{St. Dev.} & \multicolumn{1}{c}{Min} & \multicolumn{1}{c}{Max} \\ 
\hline \\[-1.8ex] 
Insurance Premium & 50,000 & 0.50 & 0.50 & 0 & 1 \\ 
Risk Preference & 50,000 & 224.98 & 11.17 & 196.93 & 256.46 \\ 
\hline \\[-1.8ex] 
\end{tabular} 
\end{table}

\subsection{5. Regression results}\label{regression-results}

\begin{Shaded}
\begin{Highlighting}[]
\FunctionTok{library}\NormalTok{(lmtest)}
\end{Highlighting}
\end{Shaded}

\begin{verbatim}
Loading required package: zoo
\end{verbatim}

\begin{verbatim}

Attaching package: 'zoo'
\end{verbatim}

\begin{verbatim}
The following objects are masked from 'package:base':

    as.Date, as.Date.numeric
\end{verbatim}

\begin{Shaded}
\begin{Highlighting}[]
\FunctionTok{library}\NormalTok{(plm)}
\FunctionTok{library}\NormalTok{(sandwich)}
\FunctionTok{library}\NormalTok{(stargazer)}

\NormalTok{model1 }\OtherTok{\textless{}{-}} \FunctionTok{lm}\NormalTok{(insprem }\SpecialCharTok{\textasciitilde{}}\NormalTok{ activity\_level, }\AttributeTok{data =}\NormalTok{ data)}
\NormalTok{model2 }\OtherTok{\textless{}{-}} \FunctionTok{lm}\NormalTok{(insprem }\SpecialCharTok{\textasciitilde{}}\NormalTok{ activity\_level }\SpecialCharTok{+}\NormalTok{ risk\_pref, }\AttributeTok{data =}\NormalTok{ data)}

\FunctionTok{stargazer}\NormalTok{(model1, model2,}
  \AttributeTok{type =} \StringTok{"latex"}\NormalTok{,}
  \AttributeTok{title =} \StringTok{"Effect of Activity Level on Insurance Premium"}\NormalTok{,}
  \AttributeTok{dep.var.labels =} \FunctionTok{c}\NormalTok{(}\StringTok{"Insurance Premium"}\NormalTok{),}
  \AttributeTok{covariate.labels =} \FunctionTok{c}\NormalTok{(}\StringTok{"Activity Level"}\NormalTok{, }\StringTok{"Risk Preference"}\NormalTok{),}
  \AttributeTok{omit.stat =} \FunctionTok{c}\NormalTok{(}\StringTok{"f"}\NormalTok{, }\StringTok{"ser"}\NormalTok{),}
  \AttributeTok{no.space =} \ConstantTok{TRUE}\NormalTok{)}
\end{Highlighting}
\end{Shaded}

\% Table created by stargazer v.5.2.3 by Marek Hlavac, Social Policy
Institute. E-mail: marek.hlavac at gmail.com \% Date and time: Thu, Sep
04, 2025 - 13:00:16

\begin{table}[!htbp] \centering 
  \caption{Effect of Activity Level on Insurance Premium} 
  \label{} 
\begin{tabular}{@{\extracolsep{5pt}}lcc} 
\\[-1.8ex]\hline 
\hline \\[-1.8ex] 
 & \multicolumn{2}{c}{\textit{Dependent variable:}} \\ 
\cline{2-3} 
\\[-1.8ex] & \multicolumn{2}{c}{Insurance Premium} \\ 
\\[-1.8ex] & (1) & (2)\\ 
\hline \\[-1.8ex] 
 Activity Level & 4.978$^{***}$ & 5.000$^{***}$ \\ 
  & (0.045) & (0.000) \\ 
  Risk Preference &  & 20.000$^{***}$ \\ 
  &  & (0.000) \\ 
  Constant & 210.054$^{***}$ & 200.000$^{***}$ \\ 
  & (0.141) & (0.000) \\ 
 \hline \\[-1.8ex] 
Observations & 50,000 & 50,000 \\ 
R$^{2}$ & 0.198 & 1.000 \\ 
Adjusted R$^{2}$ & 0.198 & 1.000 \\ 
\hline 
\hline \\[-1.8ex] 
\textit{Note:}  & \multicolumn{2}{r}{$^{*}$p$<$0.1; $^{**}$p$<$0.05; $^{***}$p$<$0.01} \\ 
\end{tabular} 
\end{table}

Putting perfect collinearity aside, activity level is significant in
both regressions, with it being slightly higher in the fixed effects
regression. Risk preference is also significant in the fixed effects
regression. The fixed effects regression likely provides more accuarate
estimates due to addressing endogeneity concerns, as risk preference is
a confounding variable that affects both activity level and insurance
premium.

I must have coded the data wrong given the perfect collinearity issue,
but I am not sure where the error is. Feedback would be appreciated.

\section{IV Model}\label{iv-model}

\subsection{1. Research Question}\label{research-question-1}

How does exercise frequency affect mental health, using weather as an
instrument?

\subsection{2. Dataset}\label{dataset}

\begin{Shaded}
\begin{Highlighting}[]
\FunctionTok{library}\NormalTok{(truncnorm)}
\FunctionTok{set.seed}\NormalTok{(}\DecValTok{0219}\NormalTok{)}
\NormalTok{n }\OtherTok{\textless{}{-}} \DecValTok{1000}
\NormalTok{weather }\OtherTok{\textless{}{-}} \FunctionTok{rbinom}\NormalTok{(n, }\DecValTok{1}\NormalTok{, }\FloatTok{0.5}\NormalTok{)}
\NormalTok{socializing }\OtherTok{\textless{}{-}} \FunctionTok{rtruncnorm}\NormalTok{(n, }\AttributeTok{a =} \DecValTok{0}\NormalTok{, }\AttributeTok{b =} \DecValTok{30}\NormalTok{, }\AttributeTok{mean =} \DecValTok{10}\NormalTok{, }\AttributeTok{sd =} \DecValTok{2}\NormalTok{)}
\NormalTok{motivation }\OtherTok{\textless{}{-}} \FunctionTok{rtruncnorm}\NormalTok{(n, }\AttributeTok{a =} \DecValTok{0}\NormalTok{, }\AttributeTok{b =} \DecValTok{10}\NormalTok{, }\AttributeTok{mean =} \DecValTok{5}\NormalTok{, }\AttributeTok{sd =} \DecValTok{1}\NormalTok{)}
\NormalTok{exercise\_freq }\OtherTok{\textless{}{-}} \DecValTok{5} \SpecialCharTok{+} \DecValTok{3} \SpecialCharTok{*}\NormalTok{ weather }\SpecialCharTok{+} \FloatTok{0.5} \SpecialCharTok{*}\NormalTok{ socializing }\SpecialCharTok{+} \FloatTok{0.2} \SpecialCharTok{*}\NormalTok{ motivation }\SpecialCharTok{+} \FunctionTok{rnorm}\NormalTok{(n, }\AttributeTok{mean =} \DecValTok{0}\NormalTok{, }\AttributeTok{sd =} \DecValTok{2}\NormalTok{)}
\NormalTok{mental\_health\_true }\OtherTok{\textless{}{-}} \DecValTok{50} \SpecialCharTok{+} \DecValTok{2} \SpecialCharTok{*}\NormalTok{ exercise\_freq }\SpecialCharTok{+} \FloatTok{1.5} \SpecialCharTok{*}\NormalTok{ socializing }\SpecialCharTok{+} \DecValTok{3} \SpecialCharTok{*}\NormalTok{ motivation }\SpecialCharTok{+} \FunctionTok{rnorm}\NormalTok{(n, }\AttributeTok{mean =} \DecValTok{0}\NormalTok{, }\AttributeTok{sd =} \DecValTok{5}\NormalTok{)}
\NormalTok{mental\_health\_obs }\OtherTok{\textless{}{-}} \DecValTok{50} \SpecialCharTok{+} \DecValTok{2} \SpecialCharTok{*}\NormalTok{ exercise\_freq }\SpecialCharTok{+} \FloatTok{1.5} \SpecialCharTok{*}\NormalTok{ socializing }\SpecialCharTok{+} \FunctionTok{rnorm}\NormalTok{(n, }\AttributeTok{mean =} \DecValTok{0}\NormalTok{, }\AttributeTok{sd =} \DecValTok{5}\NormalTok{)}
\NormalTok{data\_iv }\OtherTok{\textless{}{-}} \FunctionTok{data.frame}\NormalTok{(}\AttributeTok{weather =}\NormalTok{ weather,}
  \AttributeTok{exercise\_freq =}\NormalTok{ exercise\_freq,}
  \AttributeTok{socializing =}\NormalTok{ socializing,}
  \AttributeTok{motivation =}\NormalTok{ motivation,}
  \AttributeTok{mental\_health\_true =}\NormalTok{ mental\_health\_true,}
  \AttributeTok{mental\_health\_obs =}\NormalTok{ mental\_health\_obs)}
\end{Highlighting}
\end{Shaded}

The unit of observation is an individual with n = 1000. In this model,
\(Y\) is mental health score, \(X\) is exercise frequency, \(Z\) is
weather (instrument), and \(W\) is socializing frequency. The true model
is:

\[MentalHealth_i = \beta_0 + \beta_1\cdot ExerciseFreq_i + \beta_2\cdot Socializing_i + \beta_3\cdot Motivation_i + \epsilon_i,\]

where \(MentalHealth_i\) is the mental health score for individual i,
\(ExerciseFreq_i\) is the exercise frequency of individual i,
\(Socializing_i\) is the socializing frequency of individual i,
\(Motivation_i\) is the motivation level of individual i, and
\(\epsilon_i\) is the error term. The true coefficients are:
\(\beta_0 = 50\), \(\beta_1 = 2\), \(\beta_2 = 1.5\), \(\beta_3 = 3\),
and \(\sigma = 5\). We observe \(MentalHealth_i\), \(ExerciseFreq_i\),
and \(Socializing_i\), but we do not observe \(Motivation_i\), so the
model is misspecified as

\[MentalHealth_i = \alpha_0 + \alpha_1\cdot ExerciseFreq_i + \alpha_2\cdot Socializing_i + u_i,\]

where \(u_i\) is the error term which includes the motivation parameter.

\subsection{3. Regressions}\label{regressions-1}

The regression with endogeneity problem is:

\[MentalHealth_i = \alpha_0 + \alpha_1\cdot ExerciseFreq_i + \alpha_2\cdot Socializing_i + u_i,\]

where \(u_i\) is the error term which includes the motivation parameter.
The regression that ``fixes'' the endogeneity problem using IV is:

\[MentalHealth_i = \gamma_0 + \gamma_1\cdot \hat{ExerciseFreq}_i + \gamma_2\cdot Socializing_i + v_i,\]

where \(\hat{ExerciseFreq}_i\) is the predicted exercise frequency from
the first stage regression using weather as an instrument, and \(v_i\)
is the error term.

\subsection{4. Summary statistics}\label{summary-statistics-1}

\begin{Shaded}
\begin{Highlighting}[]
\FunctionTok{library}\NormalTok{(stargazer)}
\FunctionTok{stargazer}\NormalTok{(data\_iv, }
  \AttributeTok{title =} \StringTok{"Summary Statistics"}\NormalTok{, }
  \AttributeTok{type =} \StringTok{"latex"}\NormalTok{, }\AttributeTok{digits =} \DecValTok{2}\NormalTok{,}
  \AttributeTok{covariate.labels =} \FunctionTok{c}\NormalTok{(}\StringTok{"Weather"}\NormalTok{, }\StringTok{"Exercise Frequency"}\NormalTok{, }\StringTok{"Socializing Frequency"}\NormalTok{, }\StringTok{"Motivation"}\NormalTok{, }\StringTok{"Mental Health (True)"}\NormalTok{, }\StringTok{"Mental Health (Observed)"}\NormalTok{),}
  \AttributeTok{summary.stat =} \FunctionTok{c}\NormalTok{(}\StringTok{"n"}\NormalTok{, }\StringTok{"mean"}\NormalTok{, }\StringTok{"sd"}\NormalTok{, }\StringTok{"min"}\NormalTok{, }\StringTok{"max"}\NormalTok{))}
\end{Highlighting}
\end{Shaded}

\% Table created by stargazer v.5.2.3 by Marek Hlavac, Social Policy
Institute. E-mail: marek.hlavac at gmail.com \% Date and time: Thu, Sep
04, 2025 - 13:00:16

\begin{table}[!htbp] \centering 
  \caption{Summary Statistics} 
  \label{} 
\begin{tabular}{@{\extracolsep{5pt}}lccccc} 
\\[-1.8ex]\hline 
\hline \\[-1.8ex] 
Statistic & \multicolumn{1}{c}{N} & \multicolumn{1}{c}{Mean} & \multicolumn{1}{c}{St. Dev.} & \multicolumn{1}{c}{Min} & \multicolumn{1}{c}{Max} \\ 
\hline \\[-1.8ex] 
Weather & 1,000 & 0.48 & 0.50 & 0 & 1 \\ 
Exercise Frequency & 1,000 & 12.39 & 2.73 & 2.83 & 20.05 \\ 
Socializing Frequency & 1,000 & 9.96 & 2.04 & 3.84 & 15.80 \\ 
Motivation & 1,000 & 5.01 & 1.04 & 1.48 & 9.10 \\ 
Mental Health (True) & 1,000 & 104.73 & 9.76 & 78.14 & 135.67 \\ 
Mental Health (Observed) & 1,000 & 89.78 & 8.76 & 54.27 & 115.83 \\ 
\hline \\[-1.8ex] 
\end{tabular} 
\end{table}

\subsection{5. Regression results}\label{regression-results-1}

Exercise frequency is not signifcant in the IV regression, but is in the
OLS. Socializing frequency is significant in both regressions. The IV
regression likely provides more accuarate estimates due to addressing
endogeneity concerns, as weather is a valid instrument that affects
exercise frequency but is not directly related to mental health. The
results suggest that socialization is a more important factor for mental
health than exercise frequency when accounting for endogeneity.

\begin{Shaded}
\begin{Highlighting}[]
\FunctionTok{library}\NormalTok{(AER)}
\end{Highlighting}
\end{Shaded}

\begin{verbatim}
Loading required package: car
\end{verbatim}

\begin{verbatim}
Loading required package: carData
\end{verbatim}

\begin{verbatim}
Loading required package: survival
\end{verbatim}

\begin{Shaded}
\begin{Highlighting}[]
\FunctionTok{library}\NormalTok{(stargazer) }

\NormalTok{model1\_ols }\OtherTok{\textless{}{-}} \FunctionTok{lm}\NormalTok{(mental\_health\_obs }\SpecialCharTok{\textasciitilde{}}\NormalTok{ exercise\_freq, }\AttributeTok{data =}\NormalTok{ data\_iv)}
\NormalTok{model2\_ols }\OtherTok{\textless{}{-}} \FunctionTok{lm}\NormalTok{(mental\_health\_obs }\SpecialCharTok{\textasciitilde{}}\NormalTok{ exercise\_freq }\SpecialCharTok{+}\NormalTok{ socializing, }\AttributeTok{data =}\NormalTok{ data\_iv)}

\NormalTok{model1\_2sls }\OtherTok{\textless{}{-}} \FunctionTok{lm}\NormalTok{(exercise\_freq }\SpecialCharTok{\textasciitilde{}}\NormalTok{ weather, }\AttributeTok{data =}\NormalTok{ data\_iv)}
\NormalTok{data\_iv}\SpecialCharTok{$}\NormalTok{x\_now }\OtherTok{\textless{}{-}}\NormalTok{ model1\_2sls}\SpecialCharTok{$}\NormalTok{fitted.values}
\NormalTok{model2\_2sls }\OtherTok{\textless{}{-}} \FunctionTok{lm}\NormalTok{(exercise\_freq }\SpecialCharTok{\textasciitilde{}}\NormalTok{ weather }\SpecialCharTok{+}\NormalTok{ socializing, }\AttributeTok{data =}\NormalTok{ data\_iv)}
\NormalTok{data\_iv}\SpecialCharTok{$}\NormalTok{x\_w2 }\OtherTok{\textless{}{-}}\NormalTok{ model2\_2sls}\SpecialCharTok{$}\NormalTok{fitted.values}

\NormalTok{model1\_step2 }\OtherTok{\textless{}{-}} \FunctionTok{lm}\NormalTok{(mental\_health\_obs }\SpecialCharTok{\textasciitilde{}}\NormalTok{ x\_now, }\AttributeTok{data =}\NormalTok{ data\_iv)}
\NormalTok{model2\_step2 }\OtherTok{\textless{}{-}} \FunctionTok{lm}\NormalTok{(mental\_health\_obs }\SpecialCharTok{\textasciitilde{}}\NormalTok{ x\_w2 }\SpecialCharTok{+}\NormalTok{ socializing, }\AttributeTok{data =}\NormalTok{ data\_iv)}

\NormalTok{model\_iv }\OtherTok{\textless{}{-}} \FunctionTok{ivreg}\NormalTok{(mental\_health\_obs }\SpecialCharTok{\textasciitilde{}}\NormalTok{ exercise\_freq }\SpecialCharTok{+}\NormalTok{ socializing }\SpecialCharTok{|}\NormalTok{ weather }\SpecialCharTok{+}\NormalTok{ socializing, }\AttributeTok{data =}\NormalTok{ data\_iv)}

\FunctionTok{stargazer}\NormalTok{(model1\_ols, model2\_ols, model1\_step2, model2\_step2, model\_iv, }
  \AttributeTok{type =} \StringTok{"latex"}\NormalTok{,}
  \AttributeTok{title =} \StringTok{"Effect of Exercise Frequency on Mental Health"}\NormalTok{,}
  \AttributeTok{dep.var.labels =} \FunctionTok{c}\NormalTok{(}\StringTok{"Mental Health (Observed)"}\NormalTok{),}
  \AttributeTok{column.labels =} \FunctionTok{c}\NormalTok{(}\StringTok{"OLS (No W)"}\NormalTok{, }\StringTok{"OLS (With W)"}\NormalTok{, }\StringTok{"2SLS (No W)"}\NormalTok{, }\StringTok{"2SLS (With W)"}\NormalTok{, }\StringTok{"IV Regression"}\NormalTok{),}
  \AttributeTok{covariate.labels =} \FunctionTok{c}\NormalTok{(}\StringTok{"Exercise Frequency"}\NormalTok{, }\StringTok{"Socializing Frequency"}\NormalTok{, }\StringTok{"Socializing Frequency (2SLS)"}\NormalTok{, }\StringTok{"Exercise Frequency (2SLS)"}\NormalTok{),}
  \AttributeTok{omit.stat =} \FunctionTok{c}\NormalTok{(}\StringTok{"f"}\NormalTok{, }\StringTok{"ser"}\NormalTok{),}
  \AttributeTok{no.space =} \ConstantTok{TRUE}\NormalTok{,}
  \AttributeTok{flip =} \ConstantTok{TRUE}\NormalTok{,}
  \AttributeTok{float.env =} \StringTok{"sidewaystable"}\NormalTok{)}
\end{Highlighting}
\end{Shaded}

\% Table created by stargazer v.5.2.3 by Marek Hlavac, Social Policy
Institute. E-mail: marek.hlavac at gmail.com \% Date and time: Thu, Sep
04, 2025 - 13:00:17 \% Requires LaTeX packages: rotating

\begin{sidewaystable}[!htbp] \centering 
  \caption{Effect of Exercise Frequency on Mental Health} 
  \label{} 
\begin{tabular}{@{\extracolsep{5pt}}lccccc} 
\\[-1.8ex]\hline 
\hline \\[-1.8ex] 
 & \multicolumn{5}{c}{\textit{Dependent variable:}} \\ 
\cline{2-6} 
\\[-1.8ex] & \multicolumn{5}{c}{Mental Health (Observed)} \\ 
\\[-1.8ex] & \multicolumn{4}{c}{\textit{OLS}} & \textit{instrumental} \\ 
 & \multicolumn{4}{c}{\textit{}} & \textit{variable} \\ 
 & OLS (No W) & OLS (With W) & 2SLS (No W) & 2SLS (With W) & IV Regression \\ 
\\[-1.8ex] & (1) & (2) & (3) & (4) & (5)\\ 
\hline \\[-1.8ex] 
 Exercise Frequency & 2.462$^{***}$ & 2.047$^{***}$ &  &  & 1.995$^{***}$ \\ 
  & (0.065) & (0.061) &  &  & (0.101) \\ 
  Socializing Frequency &  &  &  & 1.995$^{***}$ &  \\ 
  &  &  &  & (0.133) &  \\ 
  Socializing Frequency (2SLS) &  & 1.479$^{***}$ &  & 1.505$^{***}$ & 1.505$^{***}$ \\ 
  &  & (0.082) &  & (0.120) & (0.091) \\ 
  Exercise Frequency (2SLS) &  &  & 1.945$^{***}$ &  &  \\ 
  &  &  & (0.174) &  &  \\ 
  Constant & 59.271$^{***}$ & 49.680$^{***}$ & 65.678$^{***}$ & 50.064$^{***}$ & 50.064$^{***}$ \\ 
  & (0.828) & (0.892) & (2.168) & (1.413) & (1.073) \\ 
 \hline \\[-1.8ex] 
Observations & 1,000 & 1,000 & 1,000 & 1,000 & 1,000 \\ 
R$^{2}$ & 0.588 & 0.690 & 0.112 & 0.462 & 0.690 \\ 
Adjusted R$^{2}$ & 0.587 & 0.689 & 0.111 & 0.461 & 0.689 \\ 
\hline 
\hline \\[-1.8ex] 
\textit{Note:}  & \multicolumn{5}{r}{$^{*}$p$<$0.1; $^{**}$p$<$0.05; $^{***}$p$<$0.01} \\ 
\end{tabular} 
\end{sidewaystable}




\end{document}

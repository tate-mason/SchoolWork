\documentclass{beamer}

% Theme choice
\usetheme{CambridgeUS} % professional, clean look
\usecolortheme{default} % colors can be adjusted if desired

% Packages
\usepackage[utf8]{inputenc}
\usepackage{graphicx}   % for including graphics
\usepackage{booktabs}   % for tables
\usepackage{amsmath, amssymb} % for math
\usepackage{hyperref}   % for clickable links

% Title page info
\title[Occupational Mobility]{Social Insurance and Occupational Mobility}
\subtitle{German Cubas and Pedro Silos}
\author[Tate Mason]{Tate Mason \\ \smallskip \texttt{Tate.Mason@uga.edu}}
\institute[Univ. of Georgia]{John Munro Godfrey Sr. Department of Economics \\ The University of Georgia}
\date{\today} % or custom date

% Begin document
\begin{document}

% Title page
\begin{frame}
  \titlepage
\end{frame}

% Outline
\begin{frame}{Outline}
  \tableofcontents
\end{frame}

% Section 1
\section{Introduction}

\begin{frame}{Motivation}
  \begin{itemize}
    \item Social insurance provides a cushion for workers. 
    \item Differs across countries, affecting labor market dynamics. 
    \item How can we model this effect?
  \end{itemize}
\end{frame}

\begin{frame}{Research Question}
  \begin{block}{Main Question}
    How does social insurance effect occpational experimentation?
  \end{block}
  \pause
  \begin{exampleblock}{Hypothesis}
    Providing more social insurance allows for riskier occupational choices.
  \end{exampleblock}
\end{frame}

% Section 2
\section{Methodology}

\begin{frame}{Model}
  \begin{itemize}
    \item Build upon Roy (1951) model of occupational choice.
    \item Add interaction between earnings risk, social insurance, and occupational choice.
  \end{itemize}
\end{frame}
\begin{frame}{Human Capital}
  Workers have two types of ability:
  \begin{itemize}
    \item Innate 
    \begin{itemize}
      \item occupation specific
      \item discovered through work experience
    \end{itemize}
    \pause
    \item General
    \begin{itemize}
      \item applicable to all occupations
      \item will experience occupation-specific shocks to this human capital
    \end{itemize}
  \end{itemize}
\end{frame}

\begin{frame}{Model Environment}
  Household:
  \begin{itemize}
    \item Lives for $S$ periods
    \item endowed one unit of time each period, with no leisure value
    \item workers dislike risk
    \item rank levels of consumption $c$ according to a utility function $u(c)$
  \end{itemize}
  \pause
  Labor Market:
  \begin{itemize}
    \item $J$ occupations, $j=1,...,J$
    \item workers can only work in one occupation at a time, but can switch between periods
    \item receive wage $w_j$ per unit of human capital
  \end{itemize}
\end{frame}
\begin{frame}{Value Functions}
  Value of staying in occupation $j$:
  \begin{gather*}
    V_s(\Omega_s, z, \epsilon, j) = \{u(c) + \beta\int W_{s+1}(\Omega_{s+1}, z', \epsilon', j')dF(\epsilon')\}, \\
    \shortintertext{s.t.} \\
    c = T(w_je^{\theta_j}e^ze^{\epsilon}) \\
    z' = z + \epsilon \\
    \Omega_{s+1} = \Omega_s
  \end{gather*}
\end{frame}
\begin{frame}{Value Functions, cont.}
  Value of switching to occupation $j'$:
  \begin{gather*}
    H_s(\Omega_s, \theta_{j'}, z, \epsilon, j') = \{u(c) + \beta\int W_{s+1}(\Omega_{s+1}, z', \epsilon', j')dF(\epsilon')\}, \\
    \shortintertext{s.t.} \\
    c = T(w_{j'}e^ze^{\theta_j'}e^{\epsilon_j'}e^{-c(s,\kappa)}) \\
    z' = z+ \epsilon' \\
    \Omega_{s+1} = \{\Omega_s, j', \theta_{j'}\}
  \end{gather*}
\end{frame}

\begin{frame}{Equilibrium}
  $\Psi_{j,s}(\Omega_s, z, \epsilon)$ is the distribution of workers in occupation $j$ at age $s$ with history $\Omega_s$, general human capital $z$, and shock $\epsilon$. $\Psi$ is defined for all $\Omega_s, z, \epsilon \in \mathbb{R}$\\
  \pause
  Mass of newborns:
  \begin{gather*}
    \Psi_{j,0}(\Omega_0, z, \epsilon) = \frac{1}{S}f_j \forall j\in\{1,...,J\}
  \end{gather*}
  where $f_j$ is the fraction of newborns with initial occupation specific ability $\theta_j$.
  \pause
  In period $S+1$:
  \begin{gather*}
    \Psi_{j,S+1}(\Omega_{S+1}, z, \epsilon) = 0 \forall j\in\{1,...,J\}
  \end{gather*}
  \pause
  For $s=0,...,S$:
  \begin{gather*}
    \Psi_{j,s+1}(\Omega_{s+1}, z', \epsilon) = \sum_{j'}\Psi_{j',s}(\Omega_s, z, \epsilon)I_{j,s}(j', \omega_s, \epsilon, z) \forall j\in\{1,...,J\}
  \end{gather*}
\end{frame}

\begin{frame}{Equilibrium cont.}
  Define aggregate mass of efficiency units in occupation $j$ at age $s$ as:
  \begin{gather*}
    N_j = \frac{1}{S} \sum_{s\in S}\int e^ze^{\theta_{j'}}e^{\epsilon_{j'}} + \frac{1}{S}\sum_{s\in S}\sum_{j\ne j'}\int e^{-c(s, \kappa)}d\Psi_{j,s}(\Omega_{s-1}, z, \epsilon) 
  \end{gather*}
  \pause

  This let's us define the SCE consisting of 
  \begin{enumerate}
    \item set of occupation level wages $\{w_j\}_{j=1}^J$
    \item occupational populations $\{\Psi_j\}_{j=1}^J$
    \item set of intermediate goods prices $\{p_j\}_{j=1}^J$
    \item set of occupational labor inputs $\{N_j\}_{j=1}^J$
    \item occupation-specific decision rules $\{I_{j,s}\}_{j=1,s=0}^{J,S}$
    \item value functions $\{V_s\}_{s=0}^S$
  \end{enumerate}
\end{frame}

\begin{frame}{Equilibrium Conditions}
  \begin{enumerate}
    \item Above value functions solve optimization problems
    \item Labor inputs $N_j$ are the solution to the intermediate producer optimnization problem
    \item Intermediate goods quantities $X_j$ solve the final goods producer's problem
    \item Prices $p_j$ equate supply and demand of intermediate goods
    \item Wage in occupation $j$ is the marginal product of an efficiency unit in that occupation s.t.
    \begin{gather*}
      w_j = \alpha_j N^{\alpha_j-1}_j \Pi_{j'\ne j}\{N_j'^{\alpha_j'}\}
    \end{gather*}
    \item Labor markets clear at occupational level
    \item In occupation $j$, $\Psi_j$ is the stationary distribution
    \item Final goods market clears by Walras' Law
  \end{enumerate}
\end{frame}



\begin{frame}{Data}
  \begin{itemize}
    \item Describe your dataset
    \item Number of observations, variables
    \item Time period, source
  \end{itemize}
\end{frame}


% Section 3
\section{Results}

\begin{frame}{Main Results}
  \begin{figure}
    \centering
    \includegraphics[width=0.8\textwidth]{example-figure.pdf}
    \caption{Main empirical finding}
  \end{figure}
\end{frame}

\begin{frame}{Table of Results}
  \begin{table}
    \centering
    \begin{tabular}{lcc}
      \toprule
      Variable & Coef. & Std. Error \\
      \midrule
      $X$ & 0.45 & 0.12 \\
      $Z$ & -0.23 & 0.08 \\
      \bottomrule
    \end{tabular}
    \caption{Regression results}
  \end{table}
\end{frame}

% Section 4
\section{Conclusion}

\begin{frame}{Conclusion}
  \begin{itemize}
    \item Summarize findings
    \item Contributions
    \item Future work
  \end{itemize}
\end{frame}

% References (optional)
\begin{frame}[allowframebreaks]{References}
  \bibliographystyle{apalike}
  \bibliography{references}
\end{frame}

\end{document}
